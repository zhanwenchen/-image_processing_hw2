
% Default to the notebook output style

    


% Inherit from the specified cell style.




    
\documentclass[11pt]{article}

    
    
    \usepackage[T1]{fontenc}
    % Nicer default font (+ math font) than Computer Modern for most use cases
    \usepackage{mathpazo}

    % Basic figure setup, for now with no caption control since it's done
    % automatically by Pandoc (which extracts ![](path) syntax from Markdown).
    \usepackage{graphicx}
    % We will generate all images so they have a width \maxwidth. This means
    % that they will get their normal width if they fit onto the page, but
    % are scaled down if they would overflow the margins.
    \makeatletter
    \def\maxwidth{\ifdim\Gin@nat@width>\linewidth\linewidth
    \else\Gin@nat@width\fi}
    \makeatother
    \let\Oldincludegraphics\includegraphics
    % Set max figure width to be 80% of text width, for now hardcoded.
    \renewcommand{\includegraphics}[1]{\Oldincludegraphics[width=.8\maxwidth]{#1}}
    % Ensure that by default, figures have no caption (until we provide a
    % proper Figure object with a Caption API and a way to capture that
    % in the conversion process - todo).
    \usepackage{caption}
    \DeclareCaptionLabelFormat{nolabel}{}
    \captionsetup{labelformat=nolabel}

    \usepackage{adjustbox} % Used to constrain images to a maximum size 
    \usepackage{xcolor} % Allow colors to be defined
    \usepackage{enumerate} % Needed for markdown enumerations to work
    \usepackage{geometry} % Used to adjust the document margins
    \usepackage{amsmath} % Equations
    \usepackage{amssymb} % Equations
    \usepackage{textcomp} % defines textquotesingle
    % Hack from http://tex.stackexchange.com/a/47451/13684:
    \AtBeginDocument{%
        \def\PYZsq{\textquotesingle}% Upright quotes in Pygmentized code
    }
    \usepackage{upquote} % Upright quotes for verbatim code
    \usepackage{eurosym} % defines \euro
    \usepackage[mathletters]{ucs} % Extended unicode (utf-8) support
    \usepackage[utf8x]{inputenc} % Allow utf-8 characters in the tex document
    \usepackage{fancyvrb} % verbatim replacement that allows latex
    \usepackage{grffile} % extends the file name processing of package graphics 
                         % to support a larger range 
    % The hyperref package gives us a pdf with properly built
    % internal navigation ('pdf bookmarks' for the table of contents,
    % internal cross-reference links, web links for URLs, etc.)
    \usepackage{hyperref}
    \usepackage{longtable} % longtable support required by pandoc >1.10
    \usepackage{booktabs}  % table support for pandoc > 1.12.2
    \usepackage[inline]{enumitem} % IRkernel/repr support (it uses the enumerate* environment)
    \usepackage[normalem]{ulem} % ulem is needed to support strikethroughs (\sout)
                                % normalem makes italics be italics, not underlines
    

    
    
    % Colors for the hyperref package
    \definecolor{urlcolor}{rgb}{0,.145,.698}
    \definecolor{linkcolor}{rgb}{.71,0.21,0.01}
    \definecolor{citecolor}{rgb}{.12,.54,.11}

    % ANSI colors
    \definecolor{ansi-black}{HTML}{3E424D}
    \definecolor{ansi-black-intense}{HTML}{282C36}
    \definecolor{ansi-red}{HTML}{E75C58}
    \definecolor{ansi-red-intense}{HTML}{B22B31}
    \definecolor{ansi-green}{HTML}{00A250}
    \definecolor{ansi-green-intense}{HTML}{007427}
    \definecolor{ansi-yellow}{HTML}{DDB62B}
    \definecolor{ansi-yellow-intense}{HTML}{B27D12}
    \definecolor{ansi-blue}{HTML}{208FFB}
    \definecolor{ansi-blue-intense}{HTML}{0065CA}
    \definecolor{ansi-magenta}{HTML}{D160C4}
    \definecolor{ansi-magenta-intense}{HTML}{A03196}
    \definecolor{ansi-cyan}{HTML}{60C6C8}
    \definecolor{ansi-cyan-intense}{HTML}{258F8F}
    \definecolor{ansi-white}{HTML}{C5C1B4}
    \definecolor{ansi-white-intense}{HTML}{A1A6B2}

    % commands and environments needed by pandoc snippets
    % extracted from the output of `pandoc -s`
    \providecommand{\tightlist}{%
      \setlength{\itemsep}{0pt}\setlength{\parskip}{0pt}}
    \DefineVerbatimEnvironment{Highlighting}{Verbatim}{commandchars=\\\{\}}
    % Add ',fontsize=\small' for more characters per line
    \newenvironment{Shaded}{}{}
    \newcommand{\KeywordTok}[1]{\textcolor[rgb]{0.00,0.44,0.13}{\textbf{{#1}}}}
    \newcommand{\DataTypeTok}[1]{\textcolor[rgb]{0.56,0.13,0.00}{{#1}}}
    \newcommand{\DecValTok}[1]{\textcolor[rgb]{0.25,0.63,0.44}{{#1}}}
    \newcommand{\BaseNTok}[1]{\textcolor[rgb]{0.25,0.63,0.44}{{#1}}}
    \newcommand{\FloatTok}[1]{\textcolor[rgb]{0.25,0.63,0.44}{{#1}}}
    \newcommand{\CharTok}[1]{\textcolor[rgb]{0.25,0.44,0.63}{{#1}}}
    \newcommand{\StringTok}[1]{\textcolor[rgb]{0.25,0.44,0.63}{{#1}}}
    \newcommand{\CommentTok}[1]{\textcolor[rgb]{0.38,0.63,0.69}{\textit{{#1}}}}
    \newcommand{\OtherTok}[1]{\textcolor[rgb]{0.00,0.44,0.13}{{#1}}}
    \newcommand{\AlertTok}[1]{\textcolor[rgb]{1.00,0.00,0.00}{\textbf{{#1}}}}
    \newcommand{\FunctionTok}[1]{\textcolor[rgb]{0.02,0.16,0.49}{{#1}}}
    \newcommand{\RegionMarkerTok}[1]{{#1}}
    \newcommand{\ErrorTok}[1]{\textcolor[rgb]{1.00,0.00,0.00}{\textbf{{#1}}}}
    \newcommand{\NormalTok}[1]{{#1}}
    
    % Additional commands for more recent versions of Pandoc
    \newcommand{\ConstantTok}[1]{\textcolor[rgb]{0.53,0.00,0.00}{{#1}}}
    \newcommand{\SpecialCharTok}[1]{\textcolor[rgb]{0.25,0.44,0.63}{{#1}}}
    \newcommand{\VerbatimStringTok}[1]{\textcolor[rgb]{0.25,0.44,0.63}{{#1}}}
    \newcommand{\SpecialStringTok}[1]{\textcolor[rgb]{0.73,0.40,0.53}{{#1}}}
    \newcommand{\ImportTok}[1]{{#1}}
    \newcommand{\DocumentationTok}[1]{\textcolor[rgb]{0.73,0.13,0.13}{\textit{{#1}}}}
    \newcommand{\AnnotationTok}[1]{\textcolor[rgb]{0.38,0.63,0.69}{\textbf{\textit{{#1}}}}}
    \newcommand{\CommentVarTok}[1]{\textcolor[rgb]{0.38,0.63,0.69}{\textbf{\textit{{#1}}}}}
    \newcommand{\VariableTok}[1]{\textcolor[rgb]{0.10,0.09,0.49}{{#1}}}
    \newcommand{\ControlFlowTok}[1]{\textcolor[rgb]{0.00,0.44,0.13}{\textbf{{#1}}}}
    \newcommand{\OperatorTok}[1]{\textcolor[rgb]{0.40,0.40,0.40}{{#1}}}
    \newcommand{\BuiltInTok}[1]{{#1}}
    \newcommand{\ExtensionTok}[1]{{#1}}
    \newcommand{\PreprocessorTok}[1]{\textcolor[rgb]{0.74,0.48,0.00}{{#1}}}
    \newcommand{\AttributeTok}[1]{\textcolor[rgb]{0.49,0.56,0.16}{{#1}}}
    \newcommand{\InformationTok}[1]{\textcolor[rgb]{0.38,0.63,0.69}{\textbf{\textit{{#1}}}}}
    \newcommand{\WarningTok}[1]{\textcolor[rgb]{0.38,0.63,0.69}{\textbf{\textit{{#1}}}}}
    
    
    % Define a nice break command that doesn't care if a line doesn't already
    % exist.
    \def\br{\hspace*{\fill} \\* }
    % Math Jax compatability definitions
    \def\gt{>}
    \def\lt{<}
    % Document parameters
    \title{HW2}
    
    
    

    % Pygments definitions
    
\makeatletter
\def\PY@reset{\let\PY@it=\relax \let\PY@bf=\relax%
    \let\PY@ul=\relax \let\PY@tc=\relax%
    \let\PY@bc=\relax \let\PY@ff=\relax}
\def\PY@tok#1{\csname PY@tok@#1\endcsname}
\def\PY@toks#1+{\ifx\relax#1\empty\else%
    \PY@tok{#1}\expandafter\PY@toks\fi}
\def\PY@do#1{\PY@bc{\PY@tc{\PY@ul{%
    \PY@it{\PY@bf{\PY@ff{#1}}}}}}}
\def\PY#1#2{\PY@reset\PY@toks#1+\relax+\PY@do{#2}}

\expandafter\def\csname PY@tok@w\endcsname{\def\PY@tc##1{\textcolor[rgb]{0.73,0.73,0.73}{##1}}}
\expandafter\def\csname PY@tok@c\endcsname{\let\PY@it=\textit\def\PY@tc##1{\textcolor[rgb]{0.25,0.50,0.50}{##1}}}
\expandafter\def\csname PY@tok@cp\endcsname{\def\PY@tc##1{\textcolor[rgb]{0.74,0.48,0.00}{##1}}}
\expandafter\def\csname PY@tok@k\endcsname{\let\PY@bf=\textbf\def\PY@tc##1{\textcolor[rgb]{0.00,0.50,0.00}{##1}}}
\expandafter\def\csname PY@tok@kp\endcsname{\def\PY@tc##1{\textcolor[rgb]{0.00,0.50,0.00}{##1}}}
\expandafter\def\csname PY@tok@kt\endcsname{\def\PY@tc##1{\textcolor[rgb]{0.69,0.00,0.25}{##1}}}
\expandafter\def\csname PY@tok@o\endcsname{\def\PY@tc##1{\textcolor[rgb]{0.40,0.40,0.40}{##1}}}
\expandafter\def\csname PY@tok@ow\endcsname{\let\PY@bf=\textbf\def\PY@tc##1{\textcolor[rgb]{0.67,0.13,1.00}{##1}}}
\expandafter\def\csname PY@tok@nb\endcsname{\def\PY@tc##1{\textcolor[rgb]{0.00,0.50,0.00}{##1}}}
\expandafter\def\csname PY@tok@nf\endcsname{\def\PY@tc##1{\textcolor[rgb]{0.00,0.00,1.00}{##1}}}
\expandafter\def\csname PY@tok@nc\endcsname{\let\PY@bf=\textbf\def\PY@tc##1{\textcolor[rgb]{0.00,0.00,1.00}{##1}}}
\expandafter\def\csname PY@tok@nn\endcsname{\let\PY@bf=\textbf\def\PY@tc##1{\textcolor[rgb]{0.00,0.00,1.00}{##1}}}
\expandafter\def\csname PY@tok@ne\endcsname{\let\PY@bf=\textbf\def\PY@tc##1{\textcolor[rgb]{0.82,0.25,0.23}{##1}}}
\expandafter\def\csname PY@tok@nv\endcsname{\def\PY@tc##1{\textcolor[rgb]{0.10,0.09,0.49}{##1}}}
\expandafter\def\csname PY@tok@no\endcsname{\def\PY@tc##1{\textcolor[rgb]{0.53,0.00,0.00}{##1}}}
\expandafter\def\csname PY@tok@nl\endcsname{\def\PY@tc##1{\textcolor[rgb]{0.63,0.63,0.00}{##1}}}
\expandafter\def\csname PY@tok@ni\endcsname{\let\PY@bf=\textbf\def\PY@tc##1{\textcolor[rgb]{0.60,0.60,0.60}{##1}}}
\expandafter\def\csname PY@tok@na\endcsname{\def\PY@tc##1{\textcolor[rgb]{0.49,0.56,0.16}{##1}}}
\expandafter\def\csname PY@tok@nt\endcsname{\let\PY@bf=\textbf\def\PY@tc##1{\textcolor[rgb]{0.00,0.50,0.00}{##1}}}
\expandafter\def\csname PY@tok@nd\endcsname{\def\PY@tc##1{\textcolor[rgb]{0.67,0.13,1.00}{##1}}}
\expandafter\def\csname PY@tok@s\endcsname{\def\PY@tc##1{\textcolor[rgb]{0.73,0.13,0.13}{##1}}}
\expandafter\def\csname PY@tok@sd\endcsname{\let\PY@it=\textit\def\PY@tc##1{\textcolor[rgb]{0.73,0.13,0.13}{##1}}}
\expandafter\def\csname PY@tok@si\endcsname{\let\PY@bf=\textbf\def\PY@tc##1{\textcolor[rgb]{0.73,0.40,0.53}{##1}}}
\expandafter\def\csname PY@tok@se\endcsname{\let\PY@bf=\textbf\def\PY@tc##1{\textcolor[rgb]{0.73,0.40,0.13}{##1}}}
\expandafter\def\csname PY@tok@sr\endcsname{\def\PY@tc##1{\textcolor[rgb]{0.73,0.40,0.53}{##1}}}
\expandafter\def\csname PY@tok@ss\endcsname{\def\PY@tc##1{\textcolor[rgb]{0.10,0.09,0.49}{##1}}}
\expandafter\def\csname PY@tok@sx\endcsname{\def\PY@tc##1{\textcolor[rgb]{0.00,0.50,0.00}{##1}}}
\expandafter\def\csname PY@tok@m\endcsname{\def\PY@tc##1{\textcolor[rgb]{0.40,0.40,0.40}{##1}}}
\expandafter\def\csname PY@tok@gh\endcsname{\let\PY@bf=\textbf\def\PY@tc##1{\textcolor[rgb]{0.00,0.00,0.50}{##1}}}
\expandafter\def\csname PY@tok@gu\endcsname{\let\PY@bf=\textbf\def\PY@tc##1{\textcolor[rgb]{0.50,0.00,0.50}{##1}}}
\expandafter\def\csname PY@tok@gd\endcsname{\def\PY@tc##1{\textcolor[rgb]{0.63,0.00,0.00}{##1}}}
\expandafter\def\csname PY@tok@gi\endcsname{\def\PY@tc##1{\textcolor[rgb]{0.00,0.63,0.00}{##1}}}
\expandafter\def\csname PY@tok@gr\endcsname{\def\PY@tc##1{\textcolor[rgb]{1.00,0.00,0.00}{##1}}}
\expandafter\def\csname PY@tok@ge\endcsname{\let\PY@it=\textit}
\expandafter\def\csname PY@tok@gs\endcsname{\let\PY@bf=\textbf}
\expandafter\def\csname PY@tok@gp\endcsname{\let\PY@bf=\textbf\def\PY@tc##1{\textcolor[rgb]{0.00,0.00,0.50}{##1}}}
\expandafter\def\csname PY@tok@go\endcsname{\def\PY@tc##1{\textcolor[rgb]{0.53,0.53,0.53}{##1}}}
\expandafter\def\csname PY@tok@gt\endcsname{\def\PY@tc##1{\textcolor[rgb]{0.00,0.27,0.87}{##1}}}
\expandafter\def\csname PY@tok@err\endcsname{\def\PY@bc##1{\setlength{\fboxsep}{0pt}\fcolorbox[rgb]{1.00,0.00,0.00}{1,1,1}{\strut ##1}}}
\expandafter\def\csname PY@tok@kc\endcsname{\let\PY@bf=\textbf\def\PY@tc##1{\textcolor[rgb]{0.00,0.50,0.00}{##1}}}
\expandafter\def\csname PY@tok@kd\endcsname{\let\PY@bf=\textbf\def\PY@tc##1{\textcolor[rgb]{0.00,0.50,0.00}{##1}}}
\expandafter\def\csname PY@tok@kn\endcsname{\let\PY@bf=\textbf\def\PY@tc##1{\textcolor[rgb]{0.00,0.50,0.00}{##1}}}
\expandafter\def\csname PY@tok@kr\endcsname{\let\PY@bf=\textbf\def\PY@tc##1{\textcolor[rgb]{0.00,0.50,0.00}{##1}}}
\expandafter\def\csname PY@tok@bp\endcsname{\def\PY@tc##1{\textcolor[rgb]{0.00,0.50,0.00}{##1}}}
\expandafter\def\csname PY@tok@fm\endcsname{\def\PY@tc##1{\textcolor[rgb]{0.00,0.00,1.00}{##1}}}
\expandafter\def\csname PY@tok@vc\endcsname{\def\PY@tc##1{\textcolor[rgb]{0.10,0.09,0.49}{##1}}}
\expandafter\def\csname PY@tok@vg\endcsname{\def\PY@tc##1{\textcolor[rgb]{0.10,0.09,0.49}{##1}}}
\expandafter\def\csname PY@tok@vi\endcsname{\def\PY@tc##1{\textcolor[rgb]{0.10,0.09,0.49}{##1}}}
\expandafter\def\csname PY@tok@vm\endcsname{\def\PY@tc##1{\textcolor[rgb]{0.10,0.09,0.49}{##1}}}
\expandafter\def\csname PY@tok@sa\endcsname{\def\PY@tc##1{\textcolor[rgb]{0.73,0.13,0.13}{##1}}}
\expandafter\def\csname PY@tok@sb\endcsname{\def\PY@tc##1{\textcolor[rgb]{0.73,0.13,0.13}{##1}}}
\expandafter\def\csname PY@tok@sc\endcsname{\def\PY@tc##1{\textcolor[rgb]{0.73,0.13,0.13}{##1}}}
\expandafter\def\csname PY@tok@dl\endcsname{\def\PY@tc##1{\textcolor[rgb]{0.73,0.13,0.13}{##1}}}
\expandafter\def\csname PY@tok@s2\endcsname{\def\PY@tc##1{\textcolor[rgb]{0.73,0.13,0.13}{##1}}}
\expandafter\def\csname PY@tok@sh\endcsname{\def\PY@tc##1{\textcolor[rgb]{0.73,0.13,0.13}{##1}}}
\expandafter\def\csname PY@tok@s1\endcsname{\def\PY@tc##1{\textcolor[rgb]{0.73,0.13,0.13}{##1}}}
\expandafter\def\csname PY@tok@mb\endcsname{\def\PY@tc##1{\textcolor[rgb]{0.40,0.40,0.40}{##1}}}
\expandafter\def\csname PY@tok@mf\endcsname{\def\PY@tc##1{\textcolor[rgb]{0.40,0.40,0.40}{##1}}}
\expandafter\def\csname PY@tok@mh\endcsname{\def\PY@tc##1{\textcolor[rgb]{0.40,0.40,0.40}{##1}}}
\expandafter\def\csname PY@tok@mi\endcsname{\def\PY@tc##1{\textcolor[rgb]{0.40,0.40,0.40}{##1}}}
\expandafter\def\csname PY@tok@il\endcsname{\def\PY@tc##1{\textcolor[rgb]{0.40,0.40,0.40}{##1}}}
\expandafter\def\csname PY@tok@mo\endcsname{\def\PY@tc##1{\textcolor[rgb]{0.40,0.40,0.40}{##1}}}
\expandafter\def\csname PY@tok@ch\endcsname{\let\PY@it=\textit\def\PY@tc##1{\textcolor[rgb]{0.25,0.50,0.50}{##1}}}
\expandafter\def\csname PY@tok@cm\endcsname{\let\PY@it=\textit\def\PY@tc##1{\textcolor[rgb]{0.25,0.50,0.50}{##1}}}
\expandafter\def\csname PY@tok@cpf\endcsname{\let\PY@it=\textit\def\PY@tc##1{\textcolor[rgb]{0.25,0.50,0.50}{##1}}}
\expandafter\def\csname PY@tok@c1\endcsname{\let\PY@it=\textit\def\PY@tc##1{\textcolor[rgb]{0.25,0.50,0.50}{##1}}}
\expandafter\def\csname PY@tok@cs\endcsname{\let\PY@it=\textit\def\PY@tc##1{\textcolor[rgb]{0.25,0.50,0.50}{##1}}}

\def\PYZbs{\char`\\}
\def\PYZus{\char`\_}
\def\PYZob{\char`\{}
\def\PYZcb{\char`\}}
\def\PYZca{\char`\^}
\def\PYZam{\char`\&}
\def\PYZlt{\char`\<}
\def\PYZgt{\char`\>}
\def\PYZsh{\char`\#}
\def\PYZpc{\char`\%}
\def\PYZdl{\char`\$}
\def\PYZhy{\char`\-}
\def\PYZsq{\char`\'}
\def\PYZdq{\char`\"}
\def\PYZti{\char`\~}
% for compatibility with earlier versions
\def\PYZat{@}
\def\PYZlb{[}
\def\PYZrb{]}
\makeatother


    % Exact colors from NB
    \definecolor{incolor}{rgb}{0.0, 0.0, 0.5}
    \definecolor{outcolor}{rgb}{0.545, 0.0, 0.0}



    
    % Prevent overflowing lines due to hard-to-break entities
    \sloppy 
    % Setup hyperref package
    \hypersetup{
      breaklinks=true,  % so long urls are correctly broken across lines
      colorlinks=true,
      urlcolor=urlcolor,
      linkcolor=linkcolor,
      citecolor=citecolor,
      }
    % Slightly bigger margins than the latex defaults
    
    \geometry{verbose,tmargin=1in,bmargin=1in,lmargin=1in,rmargin=1in}
    
    

    \begin{document}
    
    
    \maketitle
    
    

    
    The ``brightness'' of an image is a subjective term. The National
Telecommunications and Information Administration (NTIA)1 publishes
standard definitions for the names of attributes of signals and images.
The NTIA's US Federal Glossary of Telecommunication Terms (FS-1037C),
states that ``'brightness' should now be used only for non-quantitative
references to physiological sensations and perceptions of light.''2 So,
the term conveys no precise technical meaning. For this course, we will
define the brightness of an image to be its average intensity on a scale
of \{0, . . . , 255\} or {[}0, 1{]} depending on the class of the image.

    \begin{enumerate}
\def\labelenumi{\arabic{enumi}.}
\tightlist
\item
  Find a 24-bit truecolor image of your choice to use in these problems.
  Display it in your report with a photo credit. The image should
  exhibit a full range of values and colors. If both its linear
  dimensions are greater than 512, resize it so the the larger of the
  two dimensions is 512. Read the Matlab documentation on imresize().
  Then use it with the ``bicubic'' option to resize your image. If both
  the image's dimensions are less than or equal to 512 you may keep it
  as is. Determine the image's class and dimensions (after resizing) and
  include them in your report. Also compute the mean intensity value of
  the image using the Matlab code, mu = mean(I(:)); where, of course, I
  is the image array.
\end{enumerate}

    \begin{Verbatim}[commandchars=\\\{\}]
{\color{incolor}In [{\color{incolor}1}]:} \PY{o}{\PYZpc{}}\PY{k}{matplotlib} inline
        \PY{k+kn}{import} \PY{n+nn}{matplotlib}\PY{n+nn}{.}\PY{n+nn}{pyplot} \PY{k}{as} \PY{n+nn}{plt}
        \PY{k+kn}{import} \PY{n+nn}{matplotlib}\PY{n+nn}{.}\PY{n+nn}{image} \PY{k}{as} \PY{n+nn}{mpimg}
        \PY{k+kn}{from} \PY{n+nn}{PIL} \PY{k}{import} \PY{n}{Image}
        
        \PY{k+kn}{from} \PY{n+nn}{skimage} \PY{k}{import} \PY{n}{data}\PY{p}{,} \PY{n}{color}
        \PY{k+kn}{from} \PY{n+nn}{skimage}\PY{n+nn}{.}\PY{n+nn}{transform} \PY{k}{import} \PY{n}{rescale}\PY{p}{,} \PY{n}{resize}\PY{p}{,} \PY{n}{downscale\PYZus{}local\PYZus{}mean}
        
        \PY{k+kn}{import} \PY{n+nn}{numpy} \PY{k}{as} \PY{n+nn}{np}
\end{Verbatim}


    \begin{Verbatim}[commandchars=\\\{\}]
{\color{incolor}In [{\color{incolor}2}]:} \PY{n}{havana} \PY{o}{=} \PY{n}{Image}\PY{o}{.}\PY{n}{open}\PY{p}{(}\PY{l+s+s1}{\PYZsq{}}\PY{l+s+s1}{ZhanwenChenHavana.jpg}\PY{l+s+s1}{\PYZsq{}}\PY{p}{)}
        
        \PY{n}{maxwidth} \PY{o}{=} \PY{l+m+mi}{512}
        \PY{n}{maxheight} \PY{o}{=} \PY{l+m+mi}{512}
        \PY{n}{ratio} \PY{o}{=} \PY{n+nb}{min}\PY{p}{(}\PY{n}{maxwidth}\PY{o}{/}\PY{n}{havana}\PY{o}{.}\PY{n}{size}\PY{p}{[}\PY{l+m+mi}{0}\PY{p}{]}\PY{p}{,} \PY{n}{maxheight}\PY{o}{/}\PY{n}{havana}\PY{o}{.}\PY{n}{size}\PY{p}{[}\PY{l+m+mi}{1}\PY{p}{]}\PY{p}{)}
        \PY{n}{new\PYZus{}shape} \PY{o}{=} \PY{p}{(}\PY{n}{havana}\PY{o}{.}\PY{n}{size}\PY{p}{[}\PY{l+m+mi}{0}\PY{p}{]} \PY{o}{*} \PY{n}{ratio}\PY{p}{,} \PY{n}{havana}\PY{o}{.}\PY{n}{size}\PY{p}{[}\PY{l+m+mi}{1}\PY{p}{]} \PY{o}{*} \PY{n}{ratio}\PY{p}{)}
        \PY{n}{havana}\PY{o}{.}\PY{n}{thumbnail}\PY{p}{(}\PY{n}{new\PYZus{}shape}\PY{p}{,} \PY{n}{Image}\PY{o}{.}\PY{n}{BICUBIC}\PY{p}{)}
        \PY{n}{plt}\PY{o}{.}\PY{n}{imshow}\PY{p}{(}\PY{n}{havana}\PY{p}{)}
        \PY{n}{plt}\PY{o}{.}\PY{n}{title}\PY{p}{(}\PY{l+s+s1}{\PYZsq{}}\PY{l+s+s1}{Havana, Cuba. ©Zhanwen Chen 2016}\PY{l+s+s1}{\PYZsq{}}\PY{p}{)}
        \PY{n}{plt}\PY{o}{.}\PY{n}{show}\PY{p}{(}\PY{p}{)}
\end{Verbatim}


    \begin{center}
    \adjustimage{max size={0.9\linewidth}{0.9\paperheight}}{output_3_0.png}
    \end{center}
    { \hspace*{\fill} \\}
    
    \begin{enumerate}
\def\labelenumi{\arabic{enumi}.}
\tightlist
\item
  Determine the image's class and dimensions (after resizing) and
  include them in your report.
\end{enumerate}

    \begin{Verbatim}[commandchars=\\\{\}]
{\color{incolor}In [{\color{incolor}3}]:} \PY{n}{havana\PYZus{}np} \PY{o}{=} \PY{n}{np}\PY{o}{.}\PY{n}{array}\PY{p}{(}\PY{n}{havana}\PY{p}{)} \PY{c+c1}{\PYZsh{} }
        \PY{n+nb}{print}\PY{p}{(}\PY{l+s+s1}{\PYZsq{}}\PY{l+s+s1}{1. havana\PYZus{}np.dtype =}\PY{l+s+s1}{\PYZsq{}}\PY{p}{,} \PY{n}{havana\PYZus{}np}\PY{o}{.}\PY{n}{dtype}\PY{p}{)}
\end{Verbatim}


    \begin{Verbatim}[commandchars=\\\{\}]
1. havana\_np.dtype = uint8

    \end{Verbatim}

    \begin{enumerate}
\def\labelenumi{\arabic{enumi}.}
\tightlist
\item
  Also compute the mean intensity value of the image using the Matlab
  code, mu = mean(I(:)); where, of course, I is the image array.
\end{enumerate}

    \begin{Verbatim}[commandchars=\\\{\}]
{\color{incolor}In [{\color{incolor}4}]:} \PY{n}{original\PYZus{}mean} \PY{o}{=} \PY{n}{havana\PYZus{}np}\PY{o}{.}\PY{n}{mean}\PY{p}{(}\PY{p}{)}
        \PY{n+nb}{print}\PY{p}{(}\PY{l+s+s1}{\PYZsq{}}\PY{l+s+s1}{1. The mean intensity value of the image is}\PY{l+s+s1}{\PYZsq{}}\PY{p}{,} \PY{n}{original\PYZus{}mean}\PY{p}{)}
\end{Verbatim}


    \begin{Verbatim}[commandchars=\\\{\}]
1. The mean intensity value of the image is 128.3995238422532

    \end{Verbatim}

    \begin{enumerate}
\def\labelenumi{\arabic{enumi}.}
\setcounter{enumi}{1}
\item
\end{enumerate}

Write a Matlab function to alter the brightness of an image of class
uint8. The function should work on either grayscale or truecolor images.
It does not need to work on colormapped images.

The average intensity of an image can be used as an indicator of its
perceived brightness. Many off-the-shelf image processing programs have
a brightness control based on a percent change in image brightness.
Therefore the function you write should input an image array, say I, of
class uint8 and a number, p.~If p is less than 100 the result should be
a darker image; if it is greater than 100 the result should be brighter
than the original.

The output should be an image, say J, of class uint8 whose mean
intensity is (in theory) p percent of the mean value of I.

The first thing your function should do is convert the image to double.
Then compute the mean of the image, mu = mean(I(:));.

Then compute the brightness shift, g = (p/100 - 1)*mu;

Add g to every pixel in the image and convert the resulting image to
uint8. (That can be done without loops in one line of code.)

Return the brightness-shifted, class uint8 image.

This function should not be long, so include the code in your report
here, rather than in an appendix. If you are using Matlab 's code
sections and publish function just cut and paste the function's code
into the section and comment every line of it.

    \begin{Verbatim}[commandchars=\\\{\}]
{\color{incolor}In [{\color{incolor}5}]:} \PY{k}{def} \PY{n+nf}{adjustBrightness}\PY{p}{(}\PY{n}{image}\PY{p}{,} \PY{n}{p}\PY{p}{)}\PY{p}{:}
            \PY{n}{mean} \PY{o}{=} \PY{n}{image}\PY{o}{.}\PY{n}{mean}\PY{p}{(}\PY{p}{)}
            \PY{n}{brightness\PYZus{}shift} \PY{o}{=} \PY{p}{(}\PY{n}{p}\PY{o}{/}\PY{l+m+mf}{100.0} \PY{o}{\PYZhy{}} \PY{l+m+mf}{1.0}\PY{p}{)} \PY{o}{*} \PY{n}{mean}\PY{p}{;}
            \PY{k}{return} \PY{n}{np}\PY{o}{.}\PY{n}{rint}\PY{p}{(}\PY{n}{np}\PY{o}{.}\PY{n}{add}\PY{p}{(}\PY{n}{image}\PY{p}{,} \PY{n}{brightness\PYZus{}shift}\PY{p}{)}\PY{o}{.}\PY{n}{clip}\PY{p}{(}\PY{l+m+mi}{0}\PY{p}{,} \PY{l+m+mi}{255}\PY{p}{)}\PY{p}{)}\PY{o}{.}\PY{n}{astype}\PY{p}{(}\PY{l+s+s1}{\PYZsq{}}\PY{l+s+s1}{uint8}\PY{l+s+s1}{\PYZsq{}}\PY{p}{)} \PY{c+c1}{\PYZsh{} must clip: otherwise 256=\PYZgt{}1 instead of 255}
\end{Verbatim}


    \begin{enumerate}
\def\labelenumi{\arabic{enumi}.}
\setcounter{enumi}{2}
\item
\end{enumerate}

Increase and decrease the brightness of your image by 10\% and by 50\%.
Then you will have 4 new images: one 10\% dimmer image, one 10\%
brighter image, one 50\% dimmer image, and one 50\% brighter image.

Display these in your report, each with a caption indicating which one
it is. If you were to look at the 10\% brightness-shifted images and the
original separately in random order, viewing other images between them,
do you think you could accurately tell which was which? What about the
50\% shifted images?

    \begin{Verbatim}[commandchars=\\\{\}]
{\color{incolor}In [{\color{incolor}6}]:} \PY{n}{havana\PYZus{}darker\PYZus{}10} \PY{o}{=} \PY{n}{adjustBrightness}\PY{p}{(}\PY{n}{havana\PYZus{}np}\PY{p}{,} \PY{l+m+mi}{90}\PY{p}{)}
        \PY{n}{havana\PYZus{}brighter\PYZus{}10} \PY{o}{=} \PY{n}{adjustBrightness}\PY{p}{(}\PY{n}{havana\PYZus{}np}\PY{p}{,} \PY{l+m+mi}{110}\PY{p}{)}
        \PY{n}{havana\PYZus{}darker\PYZus{}50} \PY{o}{=} \PY{n}{adjustBrightness}\PY{p}{(}\PY{n}{havana\PYZus{}np}\PY{p}{,} \PY{l+m+mi}{50}\PY{p}{)}
        \PY{n}{havana\PYZus{}brighter\PYZus{}50} \PY{o}{=} \PY{n}{adjustBrightness}\PY{p}{(}\PY{n}{havana\PYZus{}np}\PY{p}{,} \PY{l+m+mi}{150}\PY{p}{)}
\end{Verbatim}


    \begin{Verbatim}[commandchars=\\\{\}]
{\color{incolor}In [{\color{incolor}7}]:} \PY{c+c1}{\PYZsh{} Display the original for comparison}
        \PY{n}{plt}\PY{o}{.}\PY{n}{imshow}\PY{p}{(}\PY{n}{havana}\PY{p}{)}
        \PY{n}{plt}\PY{o}{.}\PY{n}{title}\PY{p}{(}\PY{l+s+s1}{\PYZsq{}}\PY{l+s+s1}{3. Before Brightness Shift}\PY{l+s+s1}{\PYZsq{}}\PY{p}{)}
        \PY{n}{plt}\PY{o}{.}\PY{n}{show}\PY{p}{(}\PY{p}{)}
        
        \PY{c+c1}{\PYZsh{} Display the bright\PYZhy{}shifted images}
        \PY{n}{fig}\PY{p}{,} \PY{n}{axes} \PY{o}{=} \PY{n}{plt}\PY{o}{.}\PY{n}{subplots}\PY{p}{(}\PY{l+m+mi}{2}\PY{p}{,} \PY{l+m+mi}{2}\PY{p}{,} \PY{n}{figsize}\PY{o}{=}\PY{p}{(}\PY{l+m+mi}{12}\PY{p}{,}\PY{l+m+mi}{8}\PY{p}{)}\PY{p}{)}
        \PY{n}{fig}\PY{o}{.}\PY{n}{suptitle}\PY{p}{(}\PY{l+s+s1}{\PYZsq{}}\PY{l+s+s1}{3.}\PY{l+s+s1}{\PYZsq{}}\PY{p}{,} \PY{n}{x}\PY{o}{=}\PY{l+m+mf}{0.1}\PY{p}{,} \PY{n}{y}\PY{o}{=}\PY{l+m+mf}{0.95}\PY{p}{,} \PY{n}{horizontalalignment}\PY{o}{=}\PY{l+s+s1}{\PYZsq{}}\PY{l+s+s1}{left}\PY{l+s+s1}{\PYZsq{}}\PY{p}{)}
        
        \PY{n}{axes}\PY{p}{[}\PY{l+m+mi}{0}\PY{p}{]}\PY{p}{[}\PY{l+m+mi}{0}\PY{p}{]}\PY{o}{.}\PY{n}{imshow}\PY{p}{(}\PY{n}{havana\PYZus{}darker\PYZus{}10}\PY{p}{)}
        \PY{n}{axes}\PY{p}{[}\PY{l+m+mi}{0}\PY{p}{]}\PY{p}{[}\PY{l+m+mi}{0}\PY{p}{]}\PY{o}{.}\PY{n}{set\PYZus{}title}\PY{p}{(}\PY{l+s+s1}{\PYZsq{}}\PY{l+s+s1}{10}\PY{l+s+si}{\PYZpc{} d}\PY{l+s+s1}{arker}\PY{l+s+s1}{\PYZsq{}}\PY{p}{)}
        
        \PY{n}{axes}\PY{p}{[}\PY{l+m+mi}{0}\PY{p}{]}\PY{p}{[}\PY{l+m+mi}{1}\PY{p}{]}\PY{o}{.}\PY{n}{imshow}\PY{p}{(}\PY{n}{havana\PYZus{}brighter\PYZus{}10}\PY{p}{)}
        \PY{n}{axes}\PY{p}{[}\PY{l+m+mi}{0}\PY{p}{]}\PY{p}{[}\PY{l+m+mi}{1}\PY{p}{]}\PY{o}{.}\PY{n}{set\PYZus{}title}\PY{p}{(}\PY{l+s+s1}{\PYZsq{}}\PY{l+s+s1}{10}\PY{l+s+s1}{\PYZpc{}}\PY{l+s+s1}{ brighter}\PY{l+s+s1}{\PYZsq{}}\PY{p}{)}
        
        \PY{n}{axes}\PY{p}{[}\PY{l+m+mi}{1}\PY{p}{]}\PY{p}{[}\PY{l+m+mi}{0}\PY{p}{]}\PY{o}{.}\PY{n}{imshow}\PY{p}{(}\PY{n}{havana\PYZus{}darker\PYZus{}50}\PY{p}{)}
        \PY{n}{axes}\PY{p}{[}\PY{l+m+mi}{1}\PY{p}{]}\PY{p}{[}\PY{l+m+mi}{0}\PY{p}{]}\PY{o}{.}\PY{n}{set\PYZus{}title}\PY{p}{(}\PY{l+s+s1}{\PYZsq{}}\PY{l+s+s1}{50}\PY{l+s+si}{\PYZpc{} d}\PY{l+s+s1}{arker}\PY{l+s+s1}{\PYZsq{}}\PY{p}{)}
        
        \PY{n}{axes}\PY{p}{[}\PY{l+m+mi}{1}\PY{p}{]}\PY{p}{[}\PY{l+m+mi}{1}\PY{p}{]}\PY{o}{.}\PY{n}{imshow}\PY{p}{(}\PY{n}{havana\PYZus{}brighter\PYZus{}50}\PY{p}{)}
        \PY{n}{axes}\PY{p}{[}\PY{l+m+mi}{1}\PY{p}{]}\PY{p}{[}\PY{l+m+mi}{1}\PY{p}{]}\PY{o}{.}\PY{n}{set\PYZus{}title}\PY{p}{(}\PY{l+s+s1}{\PYZsq{}}\PY{l+s+s1}{50}\PY{l+s+s1}{\PYZpc{}}\PY{l+s+s1}{ brighter}\PY{l+s+s1}{\PYZsq{}}\PY{p}{)}
        
        \PY{n}{plt}\PY{o}{.}\PY{n}{show}\PY{p}{(}\PY{p}{)}
\end{Verbatim}


    \begin{center}
    \adjustimage{max size={0.9\linewidth}{0.9\paperheight}}{output_12_0.png}
    \end{center}
    { \hspace*{\fill} \\}
    
    \begin{center}
    \adjustimage{max size={0.9\linewidth}{0.9\paperheight}}{output_12_1.png}
    \end{center}
    { \hspace*{\fill} \\}
    
    3

If you were to look at the 10\% brightness-shifted images and the
original separately in random order, viewing other images between them,
do you think you could accurately tell which was which? What about the
50\% shifted images?

\hypertarget{answer-i-wouldnt-be-able-to-accurately-tell-apart-the-10-darker-or-the-10-brighter-image.-however-the-50-shift-in-either-direction-is-much-more-obvious.}{%
\paragraph{Answer: I wouldn't be able to accurately tell apart the 10\%
darker or the 10\% brighter image. However, the 50\% shift in either
direction is much more
obvious.}\label{answer-i-wouldnt-be-able-to-accurately-tell-apart-the-10-darker-or-the-10-brighter-image.-however-the-50-shift-in-either-direction-is-much-more-obvious.}}

    \begin{enumerate}
\def\labelenumi{\arabic{enumi}.}
\setcounter{enumi}{3}
\item
\end{enumerate}

From the uint8 image, I, compute and display the numbers,

\begin{Shaded}
\begin{Highlighting}[]
\NormalTok{muI = mean(I(:))}
\NormalTok{muI * }\FloatTok{0.5}
\NormalTok{muI * }\FloatTok{0.9}
\NormalTok{muI * }\FloatTok{1.1}
\NormalTok{muI * }\FloatTok{1.5}
\end{Highlighting}
\end{Shaded}

Compute and display the mean intensities of the 4 brightness-shifted
uint8 images. (Be sure to label the numbers in your report.)

    \begin{Verbatim}[commandchars=\\\{\}]
{\color{incolor}In [{\color{incolor}8}]:} \PY{n+nb}{print}\PY{p}{(}\PY{l+s+s1}{\PYZsq{}}\PY{l+s+s1}{4}\PY{l+s+se}{\PYZbs{}n}\PY{l+s+s1}{\PYZsq{}}\PY{p}{)}
        \PY{n+nb}{print}\PY{p}{(}\PY{l+s+s1}{\PYZsq{}}\PY{l+s+s1}{havana\PYZus{}darker\PYZus{}10.mean() =}\PY{l+s+s1}{\PYZsq{}}\PY{p}{,} \PY{n}{havana\PYZus{}darker\PYZus{}10}\PY{o}{.}\PY{n}{mean}\PY{p}{(}\PY{p}{)}\PY{p}{,} \PY{l+s+s1}{\PYZsq{}}\PY{l+s+s1}{while 90}\PY{l+s+s1}{\PYZpc{}}\PY{l+s+s1}{ * original\PYZus{}mean =}\PY{l+s+s1}{\PYZsq{}}\PY{p}{,} \PY{l+m+mf}{0.9} \PY{o}{*} \PY{n}{original\PYZus{}mean}\PY{p}{)}
        \PY{n+nb}{print}\PY{p}{(}\PY{l+s+s1}{\PYZsq{}}\PY{l+s+s1}{havana\PYZus{}brighter\PYZus{}10.mean() =}\PY{l+s+s1}{\PYZsq{}}\PY{p}{,} \PY{n}{havana\PYZus{}brighter\PYZus{}10}\PY{o}{.}\PY{n}{mean}\PY{p}{(}\PY{p}{)}\PY{p}{,} \PY{l+s+s1}{\PYZsq{}}\PY{l+s+s1}{while 110}\PY{l+s+s1}{\PYZpc{}}\PY{l+s+s1}{ * original\PYZus{}mean =}\PY{l+s+s1}{\PYZsq{}}\PY{p}{,} \PY{l+m+mf}{1.1} \PY{o}{*} \PY{n}{original\PYZus{}mean}\PY{p}{)}
        \PY{n+nb}{print}\PY{p}{(}\PY{l+s+s1}{\PYZsq{}}\PY{l+s+s1}{havana\PYZus{}darker\PYZus{}50.mean() =}\PY{l+s+s1}{\PYZsq{}}\PY{p}{,} \PY{n}{havana\PYZus{}darker\PYZus{}50}\PY{o}{.}\PY{n}{mean}\PY{p}{(}\PY{p}{)}\PY{p}{,} \PY{l+s+s1}{\PYZsq{}}\PY{l+s+s1}{while 50}\PY{l+s+s1}{\PYZpc{}}\PY{l+s+s1}{ * original\PYZus{}mean =}\PY{l+s+s1}{\PYZsq{}}\PY{p}{,} \PY{l+m+mf}{0.5} \PY{o}{*} \PY{n}{original\PYZus{}mean}\PY{p}{)}
        \PY{n+nb}{print}\PY{p}{(}\PY{l+s+s1}{\PYZsq{}}\PY{l+s+s1}{havana\PYZus{}brighter\PYZus{}50.mean() =}\PY{l+s+s1}{\PYZsq{}}\PY{p}{,} \PY{n}{havana\PYZus{}brighter\PYZus{}50}\PY{o}{.}\PY{n}{mean}\PY{p}{(}\PY{p}{)}\PY{p}{,} \PY{l+s+s1}{\PYZsq{}}\PY{l+s+s1}{while 150}\PY{l+s+s1}{\PYZpc{}}\PY{l+s+s1}{ * original\PYZus{}mean =}\PY{l+s+s1}{\PYZsq{}}\PY{p}{,} \PY{l+m+mf}{1.5} \PY{o}{*} \PY{n}{original\PYZus{}mean}\PY{p}{)}
\end{Verbatim}


    \begin{Verbatim}[commandchars=\\\{\}]
4

havana\_darker\_10.mean() = 115.5282429893695 while 90\% * original\_mean = 115.55957145802788
havana\_brighter\_10.mean() = 141.3111730968964 while 110\% * original\_mean = 141.23947622647853
havana\_darker\_50.mean() = 69.5359065707478 while 50\% * original\_mean = 64.1997619211266
havana\_brighter\_50.mean() = 188.36524773949168 while 150\% * original\_mean = 192.59928576337978

    \end{Verbatim}

    \begin{enumerate}
\def\labelenumi{\arabic{enumi}.}
\setcounter{enumi}{3}
\item
\end{enumerate}

Are the means of the brightness shifted images equal to the
corresponding percentages of the mean of the original image?

\hypertarget{answer-they-are-not-the-same-except-in-the-case-of-the-10-darker-shift.}{%
\paragraph{Answer: They are not the same, except in the case of the 10\%
darker
shift.}\label{answer-they-are-not-the-same-except-in-the-case-of-the-10-darker-shift.}}

Depending on the image it is likely they are not the same. The following
problems will explore the reasons for that.

    \begin{enumerate}
\def\labelenumi{\arabic{enumi}.}
\setcounter{enumi}{4}
\item
\end{enumerate}

Write a new version of your brightness shift program that does not
convert the results to uint8. After shifting the brightness of the class
double image, return it with out converting it to uint8.

    \begin{Verbatim}[commandchars=\\\{\}]
{\color{incolor}In [{\color{incolor}9}]:} \PY{c+c1}{\PYZsh{} 5.}
        \PY{k}{def} \PY{n+nf}{adjustBrithness\PYZus{}double}\PY{p}{(}\PY{n}{image}\PY{p}{,} \PY{n}{p}\PY{p}{)}\PY{p}{:}
            \PY{n}{mean} \PY{o}{=} \PY{n}{image}\PY{o}{.}\PY{n}{mean}\PY{p}{(}\PY{p}{)}
            \PY{n}{brightness\PYZus{}shift} \PY{o}{=} \PY{p}{(}\PY{n}{p}\PY{o}{/}\PY{l+m+mf}{100.0} \PY{o}{\PYZhy{}} \PY{l+m+mf}{1.0}\PY{p}{)} \PY{o}{*} \PY{n}{mean}\PY{p}{;}
            \PY{k}{return} \PY{n}{np}\PY{o}{.}\PY{n}{add}\PY{p}{(}\PY{n}{image}\PY{p}{,} \PY{n}{brightness\PYZus{}shift}\PY{p}{)}
\end{Verbatim}


    \begin{enumerate}
\def\labelenumi{\arabic{enumi}.}
\setcounter{enumi}{5}
\item
\end{enumerate}

Compute and display the mean intensities of the 4 brightness-shifted
class double images. (Be sure to label the numbers in your report.)

Are the means of the brightness shifted images equal to the
corresponding percentages of the mean of the original image?

\hypertarget{answer-without-conversion-to-uint8-the-means-are-indeed-shifted-exactly.}{%
\paragraph{Answer: Without conversion to uint8, the means are indeed
shifted
exactly.}\label{answer-without-conversion-to-uint8-the-means-are-indeed-shifted-exactly.}}

    \begin{Verbatim}[commandchars=\\\{\}]
{\color{incolor}In [{\color{incolor}10}]:} \PY{n}{havana\PYZus{}darker\PYZus{}10\PYZus{}double} \PY{o}{=} \PY{n}{adjustBrithness\PYZus{}double}\PY{p}{(}\PY{n}{havana\PYZus{}np}\PY{p}{,} \PY{l+m+mi}{90}\PY{p}{)}
         \PY{n}{havana\PYZus{}brighter\PYZus{}10\PYZus{}double} \PY{o}{=} \PY{n}{adjustBrithness\PYZus{}double}\PY{p}{(}\PY{n}{havana\PYZus{}np}\PY{p}{,} \PY{l+m+mi}{110}\PY{p}{)}
         \PY{n}{havana\PYZus{}darker\PYZus{}50\PYZus{}double} \PY{o}{=} \PY{n}{adjustBrithness\PYZus{}double}\PY{p}{(}\PY{n}{havana\PYZus{}np}\PY{p}{,} \PY{l+m+mi}{50}\PY{p}{)}
         \PY{n}{havana\PYZus{}brighter\PYZus{}50\PYZus{}double} \PY{o}{=} \PY{n}{adjustBrithness\PYZus{}double}\PY{p}{(}\PY{n}{havana\PYZus{}np}\PY{p}{,} \PY{l+m+mi}{150}\PY{p}{)}
\end{Verbatim}


    \begin{Verbatim}[commandchars=\\\{\}]
{\color{incolor}In [{\color{incolor}11}]:} \PY{n+nb}{print}\PY{p}{(}\PY{l+s+s1}{\PYZsq{}}\PY{l+s+s1}{6. havana\PYZus{}darker\PYZus{}10\PYZus{}double.mean() =}\PY{l+s+s1}{\PYZsq{}}\PY{p}{,} \PY{n}{havana\PYZus{}darker\PYZus{}10\PYZus{}double}\PY{o}{.}\PY{n}{mean}\PY{p}{(}\PY{p}{)}\PY{p}{,} \PY{l+s+s1}{\PYZsq{}}\PY{l+s+s1}{while 90}\PY{l+s+s1}{\PYZpc{}}\PY{l+s+s1}{ * original\PYZus{}mean =}\PY{l+s+s1}{\PYZsq{}}\PY{p}{,} \PY{l+m+mf}{0.9} \PY{o}{*} \PY{n}{original\PYZus{}mean}\PY{p}{)}
         \PY{n+nb}{print}\PY{p}{(}\PY{l+s+s1}{\PYZsq{}}\PY{l+s+s1}{6. havana\PYZus{}brighter\PYZus{}10\PYZus{}double.mean() =}\PY{l+s+s1}{\PYZsq{}}\PY{p}{,} \PY{n}{havana\PYZus{}brighter\PYZus{}10\PYZus{}double}\PY{o}{.}\PY{n}{mean}\PY{p}{(}\PY{p}{)}\PY{p}{,} \PY{l+s+s1}{\PYZsq{}}\PY{l+s+s1}{while 110}\PY{l+s+s1}{\PYZpc{}}\PY{l+s+s1}{ * original\PYZus{}mean =}\PY{l+s+s1}{\PYZsq{}}\PY{p}{,} \PY{l+m+mf}{1.1} \PY{o}{*} \PY{n}{original\PYZus{}mean}\PY{p}{)}
         \PY{n+nb}{print}\PY{p}{(}\PY{l+s+s1}{\PYZsq{}}\PY{l+s+s1}{6. havana\PYZus{}darker\PYZus{}50\PYZus{}double.mean() =}\PY{l+s+s1}{\PYZsq{}}\PY{p}{,} \PY{n}{havana\PYZus{}darker\PYZus{}50\PYZus{}double}\PY{o}{.}\PY{n}{mean}\PY{p}{(}\PY{p}{)}\PY{p}{,} \PY{l+s+s1}{\PYZsq{}}\PY{l+s+s1}{while 50}\PY{l+s+s1}{\PYZpc{}}\PY{l+s+s1}{ * original\PYZus{}mean =}\PY{l+s+s1}{\PYZsq{}}\PY{p}{,} \PY{l+m+mf}{0.5} \PY{o}{*} \PY{n}{original\PYZus{}mean}\PY{p}{)}
         \PY{n+nb}{print}\PY{p}{(}\PY{l+s+s1}{\PYZsq{}}\PY{l+s+s1}{6. havana\PYZus{}brighter\PYZus{}50\PYZus{}double.mean() =}\PY{l+s+s1}{\PYZsq{}}\PY{p}{,} \PY{n}{havana\PYZus{}brighter\PYZus{}50\PYZus{}double}\PY{o}{.}\PY{n}{mean}\PY{p}{(}\PY{p}{)}\PY{p}{,} \PY{l+s+s1}{\PYZsq{}}\PY{l+s+s1}{while 150}\PY{l+s+s1}{\PYZpc{}}\PY{l+s+s1}{ * original\PYZus{}mean =}\PY{l+s+s1}{\PYZsq{}}\PY{p}{,} \PY{l+m+mf}{1.5} \PY{o}{*} \PY{n}{original\PYZus{}mean}\PY{p}{)}
\end{Verbatim}


    \begin{Verbatim}[commandchars=\\\{\}]
6. havana\_darker\_10\_double.mean() = 115.55957145802779 while 90\% * original\_mean = 115.55957145802788
6. havana\_brighter\_10\_double.mean() = 141.23947622647853 while 110\% * original\_mean = 141.23947622647853
6. havana\_darker\_50\_double.mean() = 64.19976192112664 while 50\% * original\_mean = 64.1997619211266
6. havana\_brighter\_50\_double.mean() = 192.59928576338 while 150\% * original\_mean = 192.59928576337978

    \end{Verbatim}

    \begin{enumerate}
\def\labelenumi{\arabic{enumi}.}
\setcounter{enumi}{6}
\item
\end{enumerate}

The difference between the means that should, theoretically, be the same
but are not is due to two phenomena: ``round-off'' errors and
''clipping.'' When a class double image is converted to class uint8, any
intensity values that are not integers are converted to integers. Try
this:

\begin{Shaded}
\begin{Highlighting}[]
\NormalTok{uint8([}\FloatTok{128} \FloatTok{128.4} \FloatTok{128.5} \FloatTok{128.6}\NormalTok{])}
\end{Highlighting}
\end{Shaded}

    \begin{Verbatim}[commandchars=\\\{\}]
{\color{incolor}In [{\color{incolor}12}]:} \PY{n}{np}\PY{o}{.}\PY{n}{rint}\PY{p}{(}\PY{n}{np}\PY{o}{.}\PY{n}{array}\PY{p}{(}\PY{p}{[}\PY{l+m+mi}{128}\PY{p}{,} \PY{l+m+mf}{128.4}\PY{p}{,} \PY{l+m+mf}{128.5}\PY{p}{,} \PY{l+m+mf}{128.6}\PY{p}{]}\PY{p}{)}\PY{p}{)}\PY{o}{.}\PY{n}{astype}\PY{p}{(}\PY{n+nb}{int}\PY{p}{)}
\end{Verbatim}


\begin{Verbatim}[commandchars=\\\{\}]
{\color{outcolor}Out[{\color{outcolor}12}]:} array([128, 128, 128, 129])
\end{Verbatim}
            
    \begin{enumerate}
\def\labelenumi{\arabic{enumi}.}
\setcounter{enumi}{6}
\item
\end{enumerate}

Based on what you see how are double values being converted to uint8?

\hypertarget{answer-matlab-rounds-double-values-to-the-nearest-integer-when-the-fractional-component-is-strictly-smaller-the-number-is-floored-if-it-is-greater-or-equal-to-0.5-the-number-is-ceilinged.}{%
\paragraph{Answer: Matlab rounds double values to the nearest integer:
when the fractional component is strictly smaller, the number is
floored; if it is greater or equal to 0.5, the number is
ceilinged.}\label{answer-matlab-rounds-double-values-to-the-nearest-integer-when-the-fractional-component-is-strictly-smaller-the-number-is-floored-if-it-is-greater-or-equal-to-0.5-the-number-is-ceilinged.}}

    \begin{enumerate}
\def\labelenumi{\arabic{enumi}.}
\setcounter{enumi}{6}
\tightlist
\item
  Now try this:
\end{enumerate}

\begin{Shaded}
\begin{Highlighting}[]
\NormalTok{uint8([-}\FloatTok{128} \FloatTok{0} \FloatTok{255} \FloatTok{383}\NormalTok{])}
\end{Highlighting}
\end{Shaded}

    \begin{Verbatim}[commandchars=\\\{\}]
{\color{incolor}In [{\color{incolor}13}]:} \PY{n}{np}\PY{o}{.}\PY{n}{array}\PY{p}{(}\PY{p}{[}\PY{o}{\PYZhy{}}\PY{l+m+mi}{128}\PY{p}{,} \PY{l+m+mi}{0}\PY{p}{,} \PY{l+m+mi}{255}\PY{p}{,} \PY{l+m+mi}{383}\PY{p}{]}\PY{p}{)}\PY{o}{.}\PY{n}{clip}\PY{p}{(}\PY{l+m+mi}{0}\PY{p}{,} \PY{l+m+mi}{255}\PY{p}{)}\PY{o}{.}\PY{n}{astype}\PY{p}{(}\PY{n+nb}{int}\PY{p}{)}
\end{Verbatim}


\begin{Verbatim}[commandchars=\\\{\}]
{\color{outcolor}Out[{\color{outcolor}13}]:} array([  0,   0, 255, 255])
\end{Verbatim}
            
    \begin{enumerate}
\def\labelenumi{\arabic{enumi}.}
\setcounter{enumi}{6}
\item
\end{enumerate}

Why, do you suppose, is this phenomenon called clipping?

\hypertarget{answer-because-the-conversion-takes-max0-number-on-one-hand-and-minnumber-255-on-the-other.}{%
\paragraph{Answer: Because the conversion takes max(0, number) on one
hand and min(number, 255) on the
other.}\label{answer-because-the-conversion-takes-max0-number-on-one-hand-and-minnumber-255-on-the-other.}}

    \begin{enumerate}
\def\labelenumi{\arabic{enumi}.}
\setcounter{enumi}{6}
\item
\end{enumerate}

If there were no clipping, the mean intensity would shift along with all
the intensities.

Let I be the original image and let \(R\) = number of rows, \(C\) =
number of columns, and \(B\) = number of bands in \(I\). Then
\(N = R * C * B\), is the total number of scalar intensity values in
\(I\). The mean of \(I\) is

\[a_0=\frac{1}{N} \sum_{b=1}^{B} \sum_{r=1}^{R} \sum_{c=1}^{C} I(r,c,b),\]

If all the intensities were shifted by d then

\[
\begin{equation} \label{eq1}
    \begin{split}
        a_1 & = \frac{1}{N} \sum_{b=1}^{B} \sum_{r=1}^{R} \sum_{c=1}^{C} (I(r,c,b)+d) \\
            & = \frac{1}{N} \sum_{b=1}^{B} \sum_{r=1}^{R} \sum_{c=1}^{C} I(r,c,b) + \frac{1}{N}Nd\\
            & = a_0+d.
    \end{split}
\end{equation}
\]

However, if clipping occurs this no longer holds. Try this:

\begin{Shaded}
\begin{Highlighting}[]
\NormalTok{>> mean(}\FloatTok{0}\NormalTok{:}\FloatTok{256}\NormalTok{)}
\NormalTok{>> mean(-}\FloatTok{128}\NormalTok{:}\FloatTok{128}\NormalTok{)}
\NormalTok{>> mean(uint8(-}\FloatTok{128}\NormalTok{:}\FloatTok{128}\NormalTok{))}
\NormalTok{>> mean(}\FloatTok{128}\NormalTok{:}\FloatTok{384}\NormalTok{)}
\NormalTok{>> mean(uint8(}\FloatTok{128}\NormalTok{:}\FloatTok{384}\NormalTok{))}
\end{Highlighting}
\end{Shaded}

    \begin{Verbatim}[commandchars=\\\{\}]
{\color{incolor}In [{\color{incolor}14}]:} \PY{n+nb}{print}\PY{p}{(}\PY{l+s+s1}{\PYZsq{}}\PY{l+s+s1}{7. np.arange(257).mean() =}\PY{l+s+s1}{\PYZsq{}}\PY{p}{,} \PY{n}{np}\PY{o}{.}\PY{n}{arange}\PY{p}{(}\PY{l+m+mi}{257}\PY{p}{)}\PY{o}{.}\PY{n}{mean}\PY{p}{(}\PY{p}{)}\PY{p}{)}
         \PY{n+nb}{print}\PY{p}{(}\PY{l+s+s1}{\PYZsq{}}\PY{l+s+s1}{7. np.arange(\PYZhy{}128,129).mean() =}\PY{l+s+s1}{\PYZsq{}}\PY{p}{,} \PY{n}{np}\PY{o}{.}\PY{n}{arange}\PY{p}{(}\PY{o}{\PYZhy{}}\PY{l+m+mi}{128}\PY{p}{,}\PY{l+m+mi}{129}\PY{p}{)}\PY{o}{.}\PY{n}{mean}\PY{p}{(}\PY{p}{)}\PY{p}{)}
         \PY{n+nb}{print}\PY{p}{(}\PY{l+s+s1}{\PYZsq{}}\PY{l+s+s1}{7. np.arange(\PYZhy{}128,129).clip(0,255).mean() =}\PY{l+s+s1}{\PYZsq{}}\PY{p}{,} \PY{n}{np}\PY{o}{.}\PY{n}{arange}\PY{p}{(}\PY{o}{\PYZhy{}}\PY{l+m+mi}{128}\PY{p}{,}\PY{l+m+mi}{129}\PY{p}{)}\PY{o}{.}\PY{n}{clip}\PY{p}{(}\PY{l+m+mi}{0}\PY{p}{,}\PY{l+m+mi}{255}\PY{p}{)}\PY{o}{.}\PY{n}{mean}\PY{p}{(}\PY{p}{)}\PY{p}{)}
         \PY{n+nb}{print}\PY{p}{(}\PY{l+s+s1}{\PYZsq{}}\PY{l+s+s1}{7. np.arange(128,385).mean() =}\PY{l+s+s1}{\PYZsq{}}\PY{p}{,} \PY{n}{np}\PY{o}{.}\PY{n}{arange}\PY{p}{(}\PY{l+m+mi}{128}\PY{p}{,}\PY{l+m+mi}{385}\PY{p}{)}\PY{o}{.}\PY{n}{mean}\PY{p}{(}\PY{p}{)}\PY{p}{)}
         \PY{n+nb}{print}\PY{p}{(}\PY{l+s+s1}{\PYZsq{}}\PY{l+s+s1}{7. np.arange(128,385).clip(0,255).mean() =}\PY{l+s+s1}{\PYZsq{}}\PY{p}{,} \PY{n}{np}\PY{o}{.}\PY{n}{arange}\PY{p}{(}\PY{l+m+mi}{128}\PY{p}{,}\PY{l+m+mi}{385}\PY{p}{)}\PY{o}{.}\PY{n}{clip}\PY{p}{(}\PY{l+m+mi}{0}\PY{p}{,}\PY{l+m+mi}{255}\PY{p}{)}\PY{o}{.}\PY{n}{mean}\PY{p}{(}\PY{p}{)}\PY{p}{)}
\end{Verbatim}


    \begin{Verbatim}[commandchars=\\\{\}]
7. np.arange(257).mean() = 128.0
7. np.arange(-128,129).mean() = 0.0
7. np.arange(-128,129).clip(0,255).mean() = 32.12451361867704
7. np.arange(128,385).mean() = 256.0
7. np.arange(128,385).clip(0,255).mean() = 223.37354085603113

    \end{Verbatim}

    \begin{enumerate}
\def\labelenumi{\arabic{enumi}.}
\setcounter{enumi}{6}
\item
\end{enumerate}

Given what you have observed, how does the mean change when the image is
clipped at 0? How does it change when it is clipped at 255?

\hypertarget{answer-when-the-image-is-clipped-at-0-the-mean-is-bigger-when-there-are-negative-numbers-after-a-shift-if-the-image-is-clipped-at-255-the-mean-is-smaller-when-there-are-intensities-above-255-after-a-shift.}{%
\paragraph{Answer: When the image is clipped at 0, the mean is bigger
when there are negative numbers after a shift; if the image is clipped
at 255, the mean is smaller when there are intensities above 255 after a
shift.}\label{answer-when-the-image-is-clipped-at-0-the-mean-is-bigger-when-there-are-negative-numbers-after-a-shift-if-the-image-is-clipped-at-255-the-mean-is-smaller-when-there-are-intensities-above-255-after-a-shift.}}

    \begin{enumerate}
\def\labelenumi{\arabic{enumi}.}
\setcounter{enumi}{7}
\item
\end{enumerate}

Operate on the 50\% shifted images to ``restore'' them to their original
brightness levels. Assuming no clipping, what percentage p should you
use to restore

\hypertarget{answer}{%
\paragraph{Answer:}\label{answer}}

\hypertarget{analytically-the-we-can-express-the-restoration-p-as-a-function-of-the-original-p}{%
\paragraph{Analytically, the we can express the restoration p as a
function of the original
p:}\label{analytically-the-we-can-express-the-restoration-p-as-a-function-of-the-original-p}}

\[
d(\mu, p) = (\frac{p}{100}-1) \times \mu.
\]

\[
\begin{equation}
    \begin{split}
        d_{01} & = d(\mu_0, p_0) = (\frac{p_0}{100}-1) \times \mu_0 \\
        \mu_1 & = \mu_0 + d_{01} \\
        d_{12} & = d(\mu_1, p_1) = (\frac{p_1}{100}-1) \times \mu_1 \\
            & = (\frac{p_1}{100}-1) \times (\mu_0 + d_0) \\
            & = (\frac{p_1}{100}-1) \times (\mu_0 + (\frac{p_0}{100}-1) \times \mu_0) \\
            & = (\frac{p_1}{100}-1) \times \frac{p_0}{100} \mu_0 \\
        \mu_2 & = \mu_1 + d_{12}.
    \end{split}
\end{equation}
\]

\hypertarget{we-want-d_1-to-cancel-out-d_0.-that-is}{%
\paragraph{\texorpdfstring{We want \(d_1\) to cancel out \(d_0\). That
is,}{We want d\_1 to cancel out d\_0. That is,}}\label{we-want-d_1-to-cancel-out-d_0.-that-is}}

\[
\begin{equation}
    \begin{split}
        \mu_2 & = \mu_0 \\
        \mu_1+d_{12} & = \mu_0 \\
        \mu_0+d_{01}+d_{12} & = \mu_0 \\
        d_{12} = -d_{01} \\
        (\frac{p_1}{100}-1) \times \frac{p_0}{100} \mu_0 & = - (\frac{p_0}{100}-1) \times \mu_0 \\
        (\frac{p_1}{100}-1) \times \frac{p_0}{100} & = 1 - \frac{p_0}{100} \\
        \frac{p_1}{100} \frac{p_0}{100} - \frac{p_0}{100} & = 1 - \frac{p_0}{100} \\
        \frac{p_1}{100} \frac{p_0}{100} & = 1 \\
        p_1 & = \frac{10000}{p_0}.
    \end{split}
\end{equation}
\]

\hypertarget{using-this-formula-to-restore}{%
\paragraph{Using this formula, to
restore}\label{using-this-formula-to-restore}}

\hypertarget{a-the-50-dimmer-image-that-is}{%
\paragraph{(a) the 50\% dimmer image, that
is,}\label{a-the-50-dimmer-image-that-is}}

\[
\begin{equation}
    \begin{split}
        p_1 & = \frac{10000}{p_0} \\
            & = \frac{10000}{50} \\
            & = 200
    \end{split}
\end{equation}
\]

\hypertarget{b-the-50-brighter-image-that-is}{%
\paragraph{(b) the 50\% brighter image, that
is,}\label{b-the-50-brighter-image-that-is}}

\[
\begin{equation}
    \begin{split}
        p_1 & = \frac{10000}{p_0} \\
            & = \frac{10000}{150} \\
            & = \frac{200}{3}
    \end{split}
\end{equation}
\]

    \begin{enumerate}
\def\labelenumi{\arabic{enumi}.}
\setcounter{enumi}{7}
\item
\end{enumerate}

Use these numbers to generate two new uint8 images. Compare the mean
values of these restored uint8 images to that of the original.

    \begin{Verbatim}[commandchars=\\\{\}]
{\color{incolor}In [{\color{incolor}15}]:} \PY{n}{havana\PYZus{}darker\PYZus{}50\PYZus{}restored} \PY{o}{=} \PY{n}{adjustBrightness}\PY{p}{(}\PY{n}{havana\PYZus{}darker\PYZus{}50}\PY{p}{,} \PY{l+m+mi}{200}\PY{p}{)}
         \PY{n}{havana\PYZus{}brighter\PYZus{}50\PYZus{}restored} \PY{o}{=} \PY{n}{adjustBrightness}\PY{p}{(}\PY{n}{havana\PYZus{}brighter\PYZus{}50}\PY{p}{,} \PY{l+m+mi}{200}\PY{o}{/}\PY{l+m+mi}{3}\PY{p}{)}
         
         \PY{n+nb}{print}\PY{p}{(}\PY{l+s+s1}{\PYZsq{}}\PY{l+s+s1}{8(a). havana\PYZus{}darker\PYZus{}50\PYZus{}double\PYZus{}restored.mean() =}\PY{l+s+s1}{\PYZsq{}}\PY{p}{,} \PY{n}{havana\PYZus{}darker\PYZus{}50\PYZus{}restored}\PY{o}{.}\PY{n}{mean}\PY{p}{(}\PY{p}{)}\PY{p}{,} \PY{l+s+s1}{\PYZsq{}}\PY{l+s+s1}{while original\PYZus{}mean =}\PY{l+s+s1}{\PYZsq{}}\PY{p}{,} \PY{n}{original\PYZus{}mean}\PY{p}{)}
         \PY{n+nb}{print}\PY{p}{(}\PY{l+s+s1}{\PYZsq{}}\PY{l+s+s1}{8(b). havana\PYZus{}brighter\PYZus{}50\PYZus{}double\PYZus{}restored.mean() =}\PY{l+s+s1}{\PYZsq{}}\PY{p}{,} \PY{n}{havana\PYZus{}brighter\PYZus{}50\PYZus{}restored}\PY{o}{.}\PY{n}{mean}\PY{p}{(}\PY{p}{)}\PY{p}{,} \PY{l+s+s1}{\PYZsq{}}\PY{l+s+s1}{while original\PYZus{}mean =}\PY{l+s+s1}{\PYZsq{}}\PY{p}{,} \PY{n}{original\PYZus{}mean}\PY{p}{)}
\end{Verbatim}


    \begin{Verbatim}[commandchars=\\\{\}]
8(a). havana\_darker\_50\_double\_restored.mean() = 139.50624312683286 while original\_mean = 128.3995238422532
8(b). havana\_brighter\_50\_double\_restored.mean() = 125.36524773949169 while original\_mean = 128.3995238422532

    \end{Verbatim}

    \begin{enumerate}
\def\labelenumi{\arabic{enumi}.}
\setcounter{enumi}{7}
\tightlist
\item
  Display the images, include titles so you know which is which, and
\end{enumerate}

    \begin{Verbatim}[commandchars=\\\{\}]
{\color{incolor}In [{\color{incolor}16}]:} \PY{n}{image\PYZus{}size} \PY{o}{=} \PY{l+m+mi}{3}
         \PY{n}{fig}\PY{p}{,} \PY{n}{axes} \PY{o}{=} \PY{n}{plt}\PY{o}{.}\PY{n}{subplots}\PY{p}{(}\PY{l+m+mi}{3}\PY{p}{,} \PY{l+m+mi}{1}\PY{p}{,} \PY{n}{figsize}\PY{o}{=}\PY{p}{(}\PY{l+m+mi}{3}\PY{o}{*}\PY{n}{image\PYZus{}size}\PY{p}{,}\PY{l+m+mi}{6}\PY{o}{*}\PY{n}{image\PYZus{}size}\PY{p}{)}\PY{p}{)}
         \PY{n}{fig}\PY{o}{.}\PY{n}{suptitle}\PY{p}{(}\PY{l+s+s1}{\PYZsq{}}\PY{l+s+s1}{8.}\PY{l+s+s1}{\PYZsq{}}\PY{p}{,} \PY{n}{x}\PY{o}{=}\PY{l+m+mf}{0.1}\PY{p}{,} \PY{n}{y}\PY{o}{=}\PY{l+m+mf}{0.95}\PY{p}{,} \PY{n}{horizontalalignment}\PY{o}{=}\PY{l+s+s1}{\PYZsq{}}\PY{l+s+s1}{left}\PY{l+s+s1}{\PYZsq{}}\PY{p}{)}
         
         \PY{n}{axes}\PY{p}{[}\PY{l+m+mi}{0}\PY{p}{]}\PY{o}{.}\PY{n}{imshow}\PY{p}{(}\PY{n}{havana\PYZus{}np}\PY{p}{,} \PY{n}{aspect}\PY{o}{=}\PY{l+s+s1}{\PYZsq{}}\PY{l+s+s1}{auto}\PY{l+s+s1}{\PYZsq{}}\PY{p}{)}
         \PY{n}{axes}\PY{p}{[}\PY{l+m+mi}{0}\PY{p}{]}\PY{o}{.}\PY{n}{set\PYZus{}title}\PY{p}{(}\PY{l+s+s1}{\PYZsq{}}\PY{l+s+s1}{original}\PY{l+s+s1}{\PYZsq{}}\PY{p}{)}
         
         \PY{n}{axes}\PY{p}{[}\PY{l+m+mi}{1}\PY{p}{]}\PY{o}{.}\PY{n}{imshow}\PY{p}{(}\PY{n}{havana\PYZus{}darker\PYZus{}50\PYZus{}restored}\PY{p}{,} \PY{n}{aspect}\PY{o}{=}\PY{l+s+s1}{\PYZsq{}}\PY{l+s+s1}{auto}\PY{l+s+s1}{\PYZsq{}}\PY{p}{)}
         \PY{n}{axes}\PY{p}{[}\PY{l+m+mi}{1}\PY{p}{]}\PY{o}{.}\PY{n}{set\PYZus{}title}\PY{p}{(}\PY{l+s+s1}{\PYZsq{}}\PY{l+s+s1}{restored from 50 percent darker}\PY{l+s+s1}{\PYZsq{}}\PY{p}{)}
         
         \PY{n}{axes}\PY{p}{[}\PY{l+m+mi}{2}\PY{p}{]}\PY{o}{.}\PY{n}{imshow}\PY{p}{(}\PY{n}{havana\PYZus{}brighter\PYZus{}50\PYZus{}restored}\PY{p}{,} \PY{n}{aspect}\PY{o}{=}\PY{l+s+s1}{\PYZsq{}}\PY{l+s+s1}{auto}\PY{l+s+s1}{\PYZsq{}}\PY{p}{)}
         \PY{n}{axes}\PY{p}{[}\PY{l+m+mi}{2}\PY{p}{]}\PY{o}{.}\PY{n}{set\PYZus{}title}\PY{p}{(}\PY{l+s+s1}{\PYZsq{}}\PY{l+s+s1}{restored from 50 percent brighter}\PY{l+s+s1}{\PYZsq{}}\PY{p}{)}
         
         
         \PY{p}{[}\PY{n}{ax}\PY{o}{.}\PY{n}{set\PYZus{}anchor}\PY{p}{(}\PY{l+s+s1}{\PYZsq{}}\PY{l+s+s1}{W}\PY{l+s+s1}{\PYZsq{}}\PY{p}{)} \PY{k}{for} \PY{n}{ax} \PY{o+ow}{in} \PY{n}{axes}\PY{p}{]} \PY{c+c1}{\PYZsh{} left align all three images}
         \PY{n}{plt}\PY{o}{.}\PY{n}{tight\PYZus{}layout}\PY{p}{(}\PY{p}{)}
         \PY{n}{plt}\PY{o}{.}\PY{n}{show}\PY{p}{(}\PY{p}{)}
\end{Verbatim}


    \begin{center}
    \adjustimage{max size={0.9\linewidth}{0.9\paperheight}}{output_35_0.png}
    \end{center}
    { \hspace*{\fill} \\}
    
    Comment on any differences you can see between them and the originals.
You do not need to include the restored images in your report. However,
if you use the Matlab publish feature to generate your report they will
be included by default. And, of course, that's OK.

\hypertarget{answer-compared-to-the-original-the-image-restored-from-the-dark-shift-seems-overall-brighter-but-with-less-contrast.-conversely-the-one-restored-from-the-bright-shift-seems-darker-on-average-but-also-with-less-contrast.}{%
\paragraph{Answer: Compared to the original, the image restored from the
dark shift seems overall brighter but with less contrast. Conversely,
the one restored from the bright shift seems darker on average but also
with less
contrast.}\label{answer-compared-to-the-original-the-image-restored-from-the-dark-shift-seems-overall-brighter-but-with-less-contrast.-conversely-the-one-restored-from-the-bright-shift-seems-darker-on-average-but-also-with-less-contrast.}}

    \begin{enumerate}
\def\labelenumi{\arabic{enumi}.}
\setcounter{enumi}{8}
\item
\end{enumerate}

If clipping occurs due to processing of any kind, that means that
information has been lost. Visually significant features of the image
may be obliterated. That's why I placed the word, restore, in quotes
above. One may be able to re-shift the mean to the original number but
the features that are lost will not return. Here, we will determine the
extent of the damage by comparing the restored images to the originals.
Compute two images that are the squared differences between the original
image and the restored 50\% brightness increased and decreased images
(the as follows:

\begin{Shaded}
\begin{Highlighting}[]
\NormalTok{>> D = (double(I) - double(J)).^}\FloatTok{2}\NormalTok{;}
\end{Highlighting}
\end{Shaded}

where I is the original uint8 image and J is one of the uint8 restored
images.

    \begin{Verbatim}[commandchars=\\\{\}]
{\color{incolor}In [{\color{incolor}17}]:} \PY{n}{D\PYZus{}darker\PYZus{}50} \PY{o}{=} \PY{n}{np}\PY{o}{.}\PY{n}{square}\PY{p}{(}\PY{n}{np}\PY{o}{.}\PY{n}{subtract}\PY{p}{(}\PY{n}{havana\PYZus{}np}\PY{o}{.}\PY{n}{astype}\PY{p}{(}\PY{n+nb}{float}\PY{p}{)}\PY{p}{,} 
                                             \PY{n}{havana\PYZus{}darker\PYZus{}50\PYZus{}restored}\PY{o}{.}\PY{n}{astype}\PY{p}{(}\PY{n+nb}{float}\PY{p}{)}\PY{p}{)}\PY{p}{)}
         
         \PY{n+nb}{print}\PY{p}{(}\PY{l+s+s1}{\PYZsq{}}\PY{l+s+s1}{9. D\PYZus{}darker\PYZus{}50.min() =}\PY{l+s+s1}{\PYZsq{}}\PY{p}{,} \PY{n}{D\PYZus{}darker\PYZus{}50}\PY{o}{.}\PY{n}{min}\PY{p}{(}\PY{p}{)}\PY{p}{,} \PY{l+s+s1}{\PYZsq{}}\PY{l+s+s1}{and D\PYZus{}darker\PYZus{}50.max() =}\PY{l+s+s1}{\PYZsq{}}\PY{p}{,} \PY{n}{D\PYZus{}darker\PYZus{}50}\PY{o}{.}\PY{n}{max}\PY{p}{(}\PY{p}{)}\PY{p}{)}
         
         \PY{n}{D\PYZus{}brighter\PYZus{}50} \PY{o}{=} \PY{n}{np}\PY{o}{.}\PY{n}{square}\PY{p}{(}\PY{n}{np}\PY{o}{.}\PY{n}{subtract}\PY{p}{(}\PY{n}{havana\PYZus{}np}\PY{o}{.}\PY{n}{astype}\PY{p}{(}\PY{n+nb}{float}\PY{p}{)}\PY{p}{,} 
                                               \PY{n}{havana\PYZus{}brighter\PYZus{}50\PYZus{}restored}\PY{o}{.}\PY{n}{astype}\PY{p}{(}\PY{n+nb}{float}\PY{p}{)}\PY{p}{)}\PY{p}{)}
         
         \PY{n+nb}{print}\PY{p}{(}\PY{l+s+s1}{\PYZsq{}}\PY{l+s+s1}{9. D\PYZus{}brighter\PYZus{}50.min() =}\PY{l+s+s1}{\PYZsq{}}\PY{p}{,} \PY{n}{D\PYZus{}brighter\PYZus{}50}\PY{o}{.}\PY{n}{min}\PY{p}{(}\PY{p}{)}\PY{p}{,} \PY{l+s+s1}{\PYZsq{}}\PY{l+s+s1}{and D\PYZus{}brighter\PYZus{}50.max() =}\PY{l+s+s1}{\PYZsq{}}\PY{p}{,} \PY{n}{D\PYZus{}brighter\PYZus{}50}\PY{o}{.}\PY{n}{max}\PY{p}{(}\PY{p}{)}\PY{p}{)}
         
         \PY{c+c1}{\PYZsh{} The 10\PYZpc{} shifts}
         \PY{n}{havana\PYZus{}darker\PYZus{}10\PYZus{}restored} \PY{o}{=} \PY{n}{adjustBrightness}\PY{p}{(}\PY{n}{havana\PYZus{}darker\PYZus{}10}\PY{p}{,} \PY{l+m+mi}{1000}\PY{o}{/}\PY{l+m+mi}{9}\PY{p}{)}
         \PY{n}{D\PYZus{}darker\PYZus{}10} \PY{o}{=} \PY{n}{np}\PY{o}{.}\PY{n}{square}\PY{p}{(}\PY{n}{np}\PY{o}{.}\PY{n}{subtract}\PY{p}{(}\PY{n}{havana\PYZus{}np}\PY{o}{.}\PY{n}{astype}\PY{p}{(}\PY{n+nb}{float}\PY{p}{)}\PY{p}{,} 
                                             \PY{n}{havana\PYZus{}darker\PYZus{}10\PYZus{}restored}\PY{o}{.}\PY{n}{astype}\PY{p}{(}\PY{n+nb}{float}\PY{p}{)}\PY{p}{)}\PY{p}{)}
         
         \PY{n+nb}{print}\PY{p}{(}\PY{l+s+s1}{\PYZsq{}}\PY{l+s+s1}{9. D\PYZus{}darker\PYZus{}10.min() =}\PY{l+s+s1}{\PYZsq{}}\PY{p}{,} \PY{n}{D\PYZus{}darker\PYZus{}10}\PY{o}{.}\PY{n}{min}\PY{p}{(}\PY{p}{)}\PY{p}{,} \PY{l+s+s1}{\PYZsq{}}\PY{l+s+s1}{and D\PYZus{}darker\PYZus{}10.max() =}\PY{l+s+s1}{\PYZsq{}}\PY{p}{,} \PY{n}{D\PYZus{}darker\PYZus{}10}\PY{o}{.}\PY{n}{max}\PY{p}{(}\PY{p}{)}\PY{p}{)}
         
         \PY{n}{havana\PYZus{}brighter\PYZus{}10\PYZus{}restored} \PY{o}{=} \PY{n}{adjustBrightness}\PY{p}{(}\PY{n}{havana\PYZus{}brighter\PYZus{}10}\PY{p}{,} \PY{l+m+mi}{1000}\PY{o}{/}\PY{l+m+mi}{11}\PY{p}{)}
         \PY{n}{D\PYZus{}brighter\PYZus{}10} \PY{o}{=} \PY{n}{np}\PY{o}{.}\PY{n}{square}\PY{p}{(}\PY{n}{np}\PY{o}{.}\PY{n}{subtract}\PY{p}{(}\PY{n}{havana\PYZus{}np}\PY{o}{.}\PY{n}{astype}\PY{p}{(}\PY{n+nb}{float}\PY{p}{)}\PY{p}{,} 
                                               \PY{n}{havana\PYZus{}brighter\PYZus{}10\PYZus{}restored}\PY{o}{.}\PY{n}{astype}\PY{p}{(}\PY{n+nb}{float}\PY{p}{)}\PY{p}{)}\PY{p}{)}
         
         \PY{n+nb}{print}\PY{p}{(}\PY{l+s+s1}{\PYZsq{}}\PY{l+s+s1}{9. D\PYZus{}brighter\PYZus{}10.min() =}\PY{l+s+s1}{\PYZsq{}}\PY{p}{,} \PY{n}{D\PYZus{}brighter\PYZus{}10}\PY{o}{.}\PY{n}{min}\PY{p}{(}\PY{p}{)}\PY{p}{,} \PY{l+s+s1}{\PYZsq{}}\PY{l+s+s1}{and D\PYZus{}brighter\PYZus{}10.max() =}\PY{l+s+s1}{\PYZsq{}}\PY{p}{,} \PY{n}{D\PYZus{}brighter\PYZus{}10}\PY{o}{.}\PY{n}{max}\PY{p}{(}\PY{p}{)}\PY{p}{)}
\end{Verbatim}


    \begin{Verbatim}[commandchars=\\\{\}]
9. D\_darker\_50.min() = 0.0 and D\_darker\_50.max() = 4900.0
9. D\_brighter\_50.min() = 0.0 and D\_brighter\_50.max() = 3969.0
9. D\_darker\_10.min() = 0.0 and D\_darker\_10.max() = 169.0
9. D\_brighter\_10.min() = 0.0 and D\_brighter\_10.max() = 169.0

    \end{Verbatim}

    \begin{enumerate}
\def\labelenumi{\arabic{enumi}.}
\setcounter{enumi}{8}
\item
\end{enumerate}

Find and display the minimum and maximum values of each difference
image. The best way to display the difference images is by using

\begin{verbatim}
>> figure
>> imagesc(D) 
>> truesize
\end{verbatim}

    \begin{Verbatim}[commandchars=\\\{\}]
{\color{incolor}In [{\color{incolor}18}]:} \PY{n}{fig}\PY{p}{,} \PY{n}{axes} \PY{o}{=} \PY{n}{plt}\PY{o}{.}\PY{n}{subplots}\PY{p}{(}\PY{l+m+mi}{2}\PY{p}{,} \PY{l+m+mi}{2}\PY{p}{,} \PY{n}{figsize}\PY{o}{=}\PY{p}{(}\PY{l+m+mi}{12}\PY{p}{,}\PY{l+m+mi}{9}\PY{p}{)}\PY{p}{)}
         \PY{n}{fig}\PY{o}{.}\PY{n}{suptitle}\PY{p}{(}\PY{l+s+s1}{\PYZsq{}}\PY{l+s+s1}{9. Difference Images}\PY{l+s+s1}{\PYZsq{}}\PY{p}{)}
         
         \PY{c+c1}{\PYZsh{} normalize our difference image to [0, 1]}
         \PY{n}{D\PYZus{}darker\PYZus{}50} \PY{o}{*}\PY{o}{=} \PY{l+m+mf}{1.0}\PY{o}{/}\PY{n}{D\PYZus{}darker\PYZus{}50}\PY{o}{.}\PY{n}{max}\PY{p}{(}\PY{p}{)}
         \PY{n}{axes}\PY{p}{[}\PY{l+m+mi}{0}\PY{p}{]}\PY{p}{[}\PY{l+m+mi}{0}\PY{p}{]}\PY{o}{.}\PY{n}{imshow}\PY{p}{(}\PY{n}{D\PYZus{}darker\PYZus{}50}\PY{p}{)}
         \PY{n}{axes}\PY{p}{[}\PY{l+m+mi}{0}\PY{p}{]}\PY{p}{[}\PY{l+m+mi}{0}\PY{p}{]}\PY{o}{.}\PY{n}{set\PYZus{}title}\PY{p}{(}\PY{l+s+s1}{\PYZsq{}}\PY{l+s+s1}{restored from darker 50}\PY{l+s+s1}{\PYZsq{}}\PY{p}{)}
         
         \PY{n}{D\PYZus{}brighter\PYZus{}50} \PY{o}{*}\PY{o}{=} \PY{l+m+mf}{1.0}\PY{o}{/}\PY{n}{D\PYZus{}brighter\PYZus{}50}\PY{o}{.}\PY{n}{max}\PY{p}{(}\PY{p}{)}
         \PY{n}{axes}\PY{p}{[}\PY{l+m+mi}{0}\PY{p}{]}\PY{p}{[}\PY{l+m+mi}{1}\PY{p}{]}\PY{o}{.}\PY{n}{imshow}\PY{p}{(}\PY{n}{D\PYZus{}brighter\PYZus{}50}\PY{p}{)}
         \PY{n}{axes}\PY{p}{[}\PY{l+m+mi}{0}\PY{p}{]}\PY{p}{[}\PY{l+m+mi}{1}\PY{p}{]}\PY{o}{.}\PY{n}{set\PYZus{}title}\PY{p}{(}\PY{l+s+s1}{\PYZsq{}}\PY{l+s+s1}{restored from brighter 50}\PY{l+s+s1}{\PYZsq{}}\PY{p}{)}
         
         \PY{n}{D\PYZus{}darker\PYZus{}10} \PY{o}{*}\PY{o}{=} \PY{l+m+mf}{1.0}\PY{o}{/}\PY{n}{D\PYZus{}darker\PYZus{}10}\PY{o}{.}\PY{n}{max}\PY{p}{(}\PY{p}{)}
         \PY{n}{axes}\PY{p}{[}\PY{l+m+mi}{1}\PY{p}{]}\PY{p}{[}\PY{l+m+mi}{0}\PY{p}{]}\PY{o}{.}\PY{n}{imshow}\PY{p}{(}\PY{n}{D\PYZus{}darker\PYZus{}10}\PY{p}{)}
         \PY{n}{axes}\PY{p}{[}\PY{l+m+mi}{1}\PY{p}{]}\PY{p}{[}\PY{l+m+mi}{0}\PY{p}{]}\PY{o}{.}\PY{n}{set\PYZus{}title}\PY{p}{(}\PY{l+s+s1}{\PYZsq{}}\PY{l+s+s1}{restored from darker 10}\PY{l+s+s1}{\PYZsq{}}\PY{p}{)}
         
         \PY{n}{D\PYZus{}brighter\PYZus{}10} \PY{o}{*}\PY{o}{=} \PY{l+m+mf}{1.0}\PY{o}{/}\PY{n}{D\PYZus{}brighter\PYZus{}10}\PY{o}{.}\PY{n}{max}\PY{p}{(}\PY{p}{)}
         \PY{n}{axes}\PY{p}{[}\PY{l+m+mi}{1}\PY{p}{]}\PY{p}{[}\PY{l+m+mi}{1}\PY{p}{]}\PY{o}{.}\PY{n}{imshow}\PY{p}{(}\PY{n}{D\PYZus{}brighter\PYZus{}10}\PY{p}{)}
         \PY{n}{axes}\PY{p}{[}\PY{l+m+mi}{1}\PY{p}{]}\PY{p}{[}\PY{l+m+mi}{1}\PY{p}{]}\PY{o}{.}\PY{n}{set\PYZus{}title}\PY{p}{(}\PY{l+s+s1}{\PYZsq{}}\PY{l+s+s1}{restored from brighter 10}\PY{l+s+s1}{\PYZsq{}}\PY{p}{)}
         \PY{n}{plt}\PY{o}{.}\PY{n}{show}\PY{p}{(}\PY{p}{)}
\end{Verbatim}


    \begin{center}
    \adjustimage{max size={0.9\linewidth}{0.9\paperheight}}{output_40_0.png}
    \end{center}
    { \hspace*{\fill} \\}
    
    \begin{enumerate}
\def\labelenumi{\arabic{enumi}.}
\setcounter{enumi}{8}
\tightlist
\item
  Comment on the results. What does each one show? How do the 10\%
  difference images compare to the 50\% difference images. Include all
  four of the difference images in your report.
\end{enumerate}

\hypertarget{answer-each-difference-image-shows-the-distance-between-the-original-image-and-the-restored-image-at-each-pixel-level.-recovering-from-darker-images-loses-the-details-in-darker-regions-and-vice-versa-for-brighter-ones.-it-is-clear-that-the-10-difference-images-are-much-less-different-than-the-50-ones.}{%
\paragraph{Answer: Each difference image shows the distance between the
original image and the restored image at each pixel level. Recovering
from darker images loses the details in darker regions, and vice versa
for brighter ones. It is clear that the 10\% difference images are much
less different than the 50\%
ones.}\label{answer-each-difference-image-shows-the-distance-between-the-original-image-and-the-restored-image-at-each-pixel-level.-recovering-from-darker-images-loses-the-details-in-darker-regions-and-vice-versa-for-brighter-ones.-it-is-clear-that-the-10-difference-images-are-much-less-different-than-the-50-ones.}}

    \begin{enumerate}
\def\labelenumi{\arabic{enumi}.}
\setcounter{enumi}{9}
\item
\end{enumerate}

Color Modification through Brightness Shifting

If the brightness of only one color band is changed, the overall color
of the image is changed.

Separate the truecolor image, I, into its three bands. That is, create
three monochrome images, IR that is the red intensity, IG that is the
green intensity, and IB that is the blue intensity from your truecolor
image.

Display each of them as grayscale images. You may need to use the Matlab
function \texttt{gray(256)} to generate the correct colormap. That is,
to display one of the images do

\begin{Shaded}
\begin{Highlighting}[]
\NormalTok{>> figure; image(IR); truesize; colormap(gray(}\FloatTok{256}\NormalTok{); }
\end{Highlighting}
\end{Shaded}

    \begin{Verbatim}[commandchars=\\\{\}]
{\color{incolor}In [{\color{incolor}19}]:} \PY{n}{IR} \PY{o}{=} \PY{n}{havana\PYZus{}np}\PY{p}{[}\PY{p}{:}\PY{p}{,}\PY{p}{:}\PY{p}{,}\PY{l+m+mi}{0}\PY{p}{]}
         \PY{n}{IG} \PY{o}{=} \PY{n}{havana\PYZus{}np}\PY{p}{[}\PY{p}{:}\PY{p}{,}\PY{p}{:}\PY{p}{,}\PY{l+m+mi}{1}\PY{p}{]}
         \PY{n}{IB} \PY{o}{=} \PY{n}{havana\PYZus{}np}\PY{p}{[}\PY{p}{:}\PY{p}{,}\PY{p}{:}\PY{p}{,}\PY{l+m+mi}{2}\PY{p}{]}
         
         \PY{n}{image\PYZus{}size} \PY{o}{=} \PY{l+m+mi}{3}
         \PY{n}{fig}\PY{p}{,} \PY{n}{axes} \PY{o}{=} \PY{n}{plt}\PY{o}{.}\PY{n}{subplots}\PY{p}{(}\PY{l+m+mi}{3}\PY{p}{,} \PY{l+m+mi}{1}\PY{p}{,} \PY{n}{figsize}\PY{o}{=}\PY{p}{(}\PY{l+m+mi}{3}\PY{o}{*}\PY{n}{image\PYZus{}size}\PY{p}{,}\PY{l+m+mi}{6}\PY{o}{*}\PY{n}{image\PYZus{}size}\PY{p}{)}\PY{p}{)}
         \PY{n}{fig}\PY{o}{.}\PY{n}{suptitle}\PY{p}{(}\PY{l+s+s1}{\PYZsq{}}\PY{l+s+s1}{10.}\PY{l+s+s1}{\PYZsq{}}\PY{p}{,} \PY{n}{x}\PY{o}{=}\PY{l+m+mf}{0.1}\PY{p}{,} \PY{n}{y}\PY{o}{=}\PY{l+m+mf}{0.95}\PY{p}{,} \PY{n}{horizontalalignment}\PY{o}{=}\PY{l+s+s1}{\PYZsq{}}\PY{l+s+s1}{left}\PY{l+s+s1}{\PYZsq{}}\PY{p}{)}
         
         \PY{n}{axes}\PY{p}{[}\PY{l+m+mi}{0}\PY{p}{]}\PY{o}{.}\PY{n}{imshow}\PY{p}{(}\PY{n}{IR}\PY{p}{,} \PY{n}{cmap}\PY{o}{=}\PY{l+s+s1}{\PYZsq{}}\PY{l+s+s1}{gray}\PY{l+s+s1}{\PYZsq{}}\PY{p}{)}
         \PY{n}{axes}\PY{p}{[}\PY{l+m+mi}{0}\PY{p}{]}\PY{o}{.}\PY{n}{set\PYZus{}title}\PY{p}{(}\PY{l+s+s1}{\PYZsq{}}\PY{l+s+s1}{red}\PY{l+s+s1}{\PYZsq{}}\PY{p}{)}
         
         \PY{n}{axes}\PY{p}{[}\PY{l+m+mi}{1}\PY{p}{]}\PY{o}{.}\PY{n}{imshow}\PY{p}{(}\PY{n}{IG}\PY{p}{,} \PY{n}{cmap}\PY{o}{=}\PY{l+s+s1}{\PYZsq{}}\PY{l+s+s1}{gray}\PY{l+s+s1}{\PYZsq{}}\PY{p}{)}
         \PY{n}{axes}\PY{p}{[}\PY{l+m+mi}{1}\PY{p}{]}\PY{o}{.}\PY{n}{set\PYZus{}title}\PY{p}{(}\PY{l+s+s1}{\PYZsq{}}\PY{l+s+s1}{green}\PY{l+s+s1}{\PYZsq{}}\PY{p}{)}
         
         \PY{n}{axes}\PY{p}{[}\PY{l+m+mi}{2}\PY{p}{]}\PY{o}{.}\PY{n}{imshow}\PY{p}{(}\PY{n}{IB}\PY{p}{,} \PY{n}{cmap}\PY{o}{=}\PY{l+s+s1}{\PYZsq{}}\PY{l+s+s1}{gray}\PY{l+s+s1}{\PYZsq{}}\PY{p}{)}
         \PY{n}{axes}\PY{p}{[}\PY{l+m+mi}{2}\PY{p}{]}\PY{o}{.}\PY{n}{set\PYZus{}title}\PY{p}{(}\PY{l+s+s1}{\PYZsq{}}\PY{l+s+s1}{blue}\PY{l+s+s1}{\PYZsq{}}\PY{p}{)}
         
         \PY{p}{[}\PY{n}{ax}\PY{o}{.}\PY{n}{set\PYZus{}anchor}\PY{p}{(}\PY{l+s+s1}{\PYZsq{}}\PY{l+s+s1}{W}\PY{l+s+s1}{\PYZsq{}}\PY{p}{)} \PY{k}{for} \PY{n}{ax} \PY{o+ow}{in} \PY{n}{axes}\PY{p}{]} \PY{c+c1}{\PYZsh{} left align all three images}
         \PY{n}{plt}\PY{o}{.}\PY{n}{tight\PYZus{}layout}\PY{p}{(}\PY{p}{)}
         \PY{n}{plt}\PY{o}{.}\PY{n}{show}\PY{p}{(}\PY{p}{)}
\end{Verbatim}


    \begin{center}
    \adjustimage{max size={0.9\linewidth}{0.9\paperheight}}{output_43_0.png}
    \end{center}
    { \hspace*{\fill} \\}
    
    \begin{enumerate}
\def\labelenumi{\arabic{enumi}.}
\setcounter{enumi}{9}
\item
\end{enumerate}

Use your brightness shifting function to:

\begin{enumerate}
\def\labelenumi{(\alph{enumi})}
\tightlist
\item
  Increase the brightness of the red band by 20\%. Then recombine it
  with the unchanged, original green and blue bands to create a new
  truecolor image. Display it and comment on its change from the
  original.
\end{enumerate}

    \begin{Verbatim}[commandchars=\\\{\}]
{\color{incolor}In [{\color{incolor}20}]:} \PY{n}{red\PYZus{}120} \PY{o}{=} \PY{n}{np}\PY{o}{.}\PY{n}{stack}\PY{p}{(}\PY{p}{[}\PY{n}{adjustBrightness}\PY{p}{(}\PY{n}{IR}\PY{p}{,} \PY{l+m+mi}{120}\PY{p}{)}\PY{p}{,} \PY{n}{IG}\PY{p}{,} \PY{n}{IB}\PY{p}{]}\PY{p}{,} \PY{n}{axis}\PY{o}{=}\PY{l+m+mi}{2}\PY{p}{)}
         \PY{n}{plt}\PY{o}{.}\PY{n}{imshow}\PY{p}{(}\PY{n}{red\PYZus{}120}\PY{p}{)}
         \PY{n}{plt}\PY{o}{.}\PY{n}{title}\PY{p}{(}\PY{l+s+s1}{\PYZsq{}}\PY{l+s+s1}{10(a). 20}\PY{l+s+si}{\PYZpc{} r}\PY{l+s+s1}{edder}\PY{l+s+s1}{\PYZsq{}}\PY{p}{)}
         \PY{n}{plt}\PY{o}{.}\PY{n}{show}\PY{p}{(}\PY{p}{)}
         \PY{n+nb}{print}\PY{p}{(}\PY{l+s+s1}{\PYZsq{}}\PY{l+s+s1}{10(a). The 20}\PY{l+s+si}{\PYZpc{} r}\PY{l+s+s1}{edder image looks warmer everywhere except where it was originally red.}\PY{l+s+s1}{\PYZsq{}}\PY{p}{)}
\end{Verbatim}


    \begin{center}
    \adjustimage{max size={0.9\linewidth}{0.9\paperheight}}{output_45_0.png}
    \end{center}
    { \hspace*{\fill} \\}
    
    \begin{Verbatim}[commandchars=\\\{\}]
10(a). The 20\% redder image looks warmer everywhere except where it was originally red.

    \end{Verbatim}

    \begin{enumerate}
\def\labelenumi{(\alph{enumi})}
\setcounter{enumi}{1}
\tightlist
\item
  Decrease the brightness of the red band by 20\%. Then recombine it
  with the unchanged, original green and blue bands to create a new
  truecolor image. Display it and comment on its change from the
  original.
\end{enumerate}

    \begin{Verbatim}[commandchars=\\\{\}]
{\color{incolor}In [{\color{incolor}21}]:} \PY{n}{red\PYZus{}80} \PY{o}{=} \PY{n}{np}\PY{o}{.}\PY{n}{stack}\PY{p}{(}\PY{p}{[}\PY{n}{adjustBrightness}\PY{p}{(}\PY{n}{IR}\PY{p}{,} \PY{l+m+mi}{80}\PY{p}{)}\PY{p}{,} \PY{n}{IG}\PY{p}{,} \PY{n}{IB}\PY{p}{]}\PY{p}{,} \PY{n}{axis}\PY{o}{=}\PY{l+m+mi}{2}\PY{p}{)}
         \PY{n}{plt}\PY{o}{.}\PY{n}{imshow}\PY{p}{(}\PY{n}{red\PYZus{}80}\PY{p}{)}
         \PY{n}{plt}\PY{o}{.}\PY{n}{title}\PY{p}{(}\PY{l+s+s1}{\PYZsq{}}\PY{l+s+s1}{10(b). 20}\PY{l+s+si}{\PYZpc{} le}\PY{l+s+s1}{ss red}\PY{l+s+s1}{\PYZsq{}}\PY{p}{)}
         \PY{n}{plt}\PY{o}{.}\PY{n}{show}\PY{p}{(}\PY{p}{)}
         \PY{n+nb}{print}\PY{p}{(}\PY{l+s+s1}{\PYZsq{}}\PY{l+s+s1}{10(b). The 20}\PY{l+s+si}{\PYZpc{} le}\PY{l+s+s1}{ss red image looks colder everywhere except where it was originally cold}\PY{l+s+s1}{\PYZsq{}}\PY{p}{)}
\end{Verbatim}


    \begin{center}
    \adjustimage{max size={0.9\linewidth}{0.9\paperheight}}{output_47_0.png}
    \end{center}
    { \hspace*{\fill} \\}
    
    \begin{Verbatim}[commandchars=\\\{\}]
10(b). The 20\% less red image looks colder everywhere except where it was originally cold

    \end{Verbatim}

    \begin{enumerate}
\def\labelenumi{(\alph{enumi})}
\setcounter{enumi}{2}
\tightlist
\item
  Increase the brightness of the green band by 20\%. Then recombine it
  with the unchanged, original red and blue bands to create a new
  truecolor image. Display it and comment on its change from the
  original.
\end{enumerate}

    \begin{Verbatim}[commandchars=\\\{\}]
{\color{incolor}In [{\color{incolor}22}]:} \PY{n}{green\PYZus{}120} \PY{o}{=} \PY{n}{np}\PY{o}{.}\PY{n}{stack}\PY{p}{(}\PY{p}{[}\PY{n}{IR}\PY{p}{,} \PY{n}{adjustBrightness}\PY{p}{(}\PY{n}{IG}\PY{p}{,} \PY{l+m+mi}{120}\PY{p}{)}\PY{p}{,} \PY{n}{IB}\PY{p}{]}\PY{p}{,} \PY{n}{axis}\PY{o}{=}\PY{l+m+mi}{2}\PY{p}{)}
         \PY{n}{plt}\PY{o}{.}\PY{n}{imshow}\PY{p}{(}\PY{n}{green\PYZus{}120}\PY{p}{)}
         \PY{n}{plt}\PY{o}{.}\PY{n}{title}\PY{p}{(}\PY{l+s+s1}{\PYZsq{}}\PY{l+s+s1}{10(c). 20}\PY{l+s+si}{\PYZpc{} g}\PY{l+s+s1}{reener}\PY{l+s+s1}{\PYZsq{}}\PY{p}{)}
         \PY{n}{plt}\PY{o}{.}\PY{n}{show}\PY{p}{(}\PY{p}{)}
         \PY{n+nb}{print}\PY{p}{(}\PY{l+s+s1}{\PYZsq{}}\PY{l+s+s1}{10(c). The 20}\PY{l+s+si}{\PYZpc{} g}\PY{l+s+s1}{reener image looks colder overall.}\PY{l+s+s1}{\PYZsq{}}\PY{p}{)}
\end{Verbatim}


    \begin{center}
    \adjustimage{max size={0.9\linewidth}{0.9\paperheight}}{output_49_0.png}
    \end{center}
    { \hspace*{\fill} \\}
    
    \begin{Verbatim}[commandchars=\\\{\}]
10(c). The 20\% greener image looks colder overall.

    \end{Verbatim}

    \begin{enumerate}
\def\labelenumi{(\alph{enumi})}
\setcounter{enumi}{3}
\tightlist
\item
  Decrease the brightness of the green band by 20\%. Then recombine it
  with the unchanged, original red and blue bands to create a new
  truecolor image. Display it and comment on its change from the
  original.
\end{enumerate}

    \begin{Verbatim}[commandchars=\\\{\}]
{\color{incolor}In [{\color{incolor}23}]:} \PY{n}{green\PYZus{}80} \PY{o}{=} \PY{n}{np}\PY{o}{.}\PY{n}{stack}\PY{p}{(}\PY{p}{[}\PY{n}{IR}\PY{p}{,} \PY{n}{adjustBrightness}\PY{p}{(}\PY{n}{IG}\PY{p}{,} \PY{l+m+mi}{80}\PY{p}{)}\PY{p}{,} \PY{n}{IB}\PY{p}{]}\PY{p}{,} \PY{n}{axis}\PY{o}{=}\PY{l+m+mi}{2}\PY{p}{)}
         \PY{n}{plt}\PY{o}{.}\PY{n}{imshow}\PY{p}{(}\PY{n}{green\PYZus{}80}\PY{p}{)}
         \PY{n}{plt}\PY{o}{.}\PY{n}{title}\PY{p}{(}\PY{l+s+s1}{\PYZsq{}}\PY{l+s+s1}{10(d). 20}\PY{l+s+si}{\PYZpc{} le}\PY{l+s+s1}{ss green}\PY{l+s+s1}{\PYZsq{}}\PY{p}{)}
         \PY{n}{plt}\PY{o}{.}\PY{n}{show}\PY{p}{(}\PY{p}{)}
         \PY{n+nb}{print}\PY{p}{(}\PY{l+s+s1}{\PYZsq{}}\PY{l+s+s1}{10(d). The 20}\PY{l+s+si}{\PYZpc{} le}\PY{l+s+s1}{ss green image looks warmer but with more blue}\PY{l+s+s1}{\PYZsq{}}\PY{p}{)}
\end{Verbatim}


    \begin{center}
    \adjustimage{max size={0.9\linewidth}{0.9\paperheight}}{output_51_0.png}
    \end{center}
    { \hspace*{\fill} \\}
    
    \begin{Verbatim}[commandchars=\\\{\}]
10(d). The 20\% less green image looks warmer but with more blue

    \end{Verbatim}

    \begin{enumerate}
\def\labelenumi{(\alph{enumi})}
\setcounter{enumi}{4}
\tightlist
\item
  Increase the brightness of the blue band by 20\%. Then recombine it
  with the unchanged, original red and green bands to create a new
  truecolor image. Display it and comment on its change from the
  original.
\end{enumerate}

    \begin{Verbatim}[commandchars=\\\{\}]
{\color{incolor}In [{\color{incolor}24}]:} \PY{n}{blue\PYZus{}120} \PY{o}{=} \PY{n}{np}\PY{o}{.}\PY{n}{stack}\PY{p}{(}\PY{p}{[}\PY{n}{IR}\PY{p}{,} \PY{n}{IG}\PY{p}{,} \PY{n}{adjustBrightness}\PY{p}{(}\PY{n}{IB}\PY{p}{,} \PY{l+m+mi}{120}\PY{p}{)}\PY{p}{]}\PY{p}{,} \PY{n}{axis}\PY{o}{=}\PY{l+m+mi}{2}\PY{p}{)}
         \PY{n}{plt}\PY{o}{.}\PY{n}{imshow}\PY{p}{(}\PY{n}{blue\PYZus{}120}\PY{p}{)}
         \PY{n}{plt}\PY{o}{.}\PY{n}{title}\PY{p}{(}\PY{l+s+s1}{\PYZsq{}}\PY{l+s+s1}{10(e). 20}\PY{l+s+s1}{\PYZpc{}}\PY{l+s+s1}{ bluer}\PY{l+s+s1}{\PYZsq{}}\PY{p}{)}
         \PY{n}{plt}\PY{o}{.}\PY{n}{show}\PY{p}{(}\PY{p}{)}
         \PY{n+nb}{print}\PY{p}{(}\PY{l+s+s1}{\PYZsq{}}\PY{l+s+s1}{10(e). The 20}\PY{l+s+s1}{\PYZpc{}}\PY{l+s+s1}{ bluer image looks colder.}\PY{l+s+s1}{\PYZsq{}}\PY{p}{)}
\end{Verbatim}


    \begin{center}
    \adjustimage{max size={0.9\linewidth}{0.9\paperheight}}{output_53_0.png}
    \end{center}
    { \hspace*{\fill} \\}
    
    \begin{Verbatim}[commandchars=\\\{\}]
10(e). The 20\% bluer image looks colder.

    \end{Verbatim}

    \begin{enumerate}
\def\labelenumi{(\alph{enumi})}
\setcounter{enumi}{5}
\tightlist
\item
  Decrease the brightness of the blue band by 20\%. Then recombine it
  with the unchanged, original red and green bands to create a new
  truecolor image. Display it and comment on its change from the
  original.
\end{enumerate}

    \begin{Verbatim}[commandchars=\\\{\}]
{\color{incolor}In [{\color{incolor}25}]:} \PY{n}{blue\PYZus{}80} \PY{o}{=} \PY{n}{np}\PY{o}{.}\PY{n}{stack}\PY{p}{(}\PY{p}{[}\PY{n}{IR}\PY{p}{,} \PY{n}{IG}\PY{p}{,} \PY{n}{adjustBrightness}\PY{p}{(}\PY{n}{IB}\PY{p}{,} \PY{l+m+mi}{80}\PY{p}{)}\PY{p}{]}\PY{p}{,} \PY{n}{axis}\PY{o}{=}\PY{l+m+mi}{2}\PY{p}{)}
         \PY{n}{plt}\PY{o}{.}\PY{n}{imshow}\PY{p}{(}\PY{n}{blue\PYZus{}80}\PY{p}{)}
         \PY{n}{plt}\PY{o}{.}\PY{n}{title}\PY{p}{(}\PY{l+s+s1}{\PYZsq{}}\PY{l+s+s1}{10(f). 20}\PY{l+s+si}{\PYZpc{} le}\PY{l+s+s1}{ss blue}\PY{l+s+s1}{\PYZsq{}}\PY{p}{)}
         \PY{n}{plt}\PY{o}{.}\PY{n}{show}\PY{p}{(}\PY{p}{)}
         \PY{n+nb}{print}\PY{p}{(}\PY{l+s+s1}{\PYZsq{}}\PY{l+s+s1}{10(f). The 20}\PY{l+s+si}{\PYZpc{} le}\PY{l+s+s1}{ss blue image also looks warmer but a little more green overall.}\PY{l+s+s1}{\PYZsq{}}\PY{p}{)}
\end{Verbatim}


    \begin{center}
    \adjustimage{max size={0.9\linewidth}{0.9\paperheight}}{output_55_0.png}
    \end{center}
    { \hspace*{\fill} \\}
    
    \begin{Verbatim}[commandchars=\\\{\}]
10(f). The 20\% less blue image also looks warmer but a little more green overall.

    \end{Verbatim}

    In general, how might you use this approach to tint the image with any
given color?

\hypertarget{answer-i-could-implement-a-function-with-p-parameters-for-each-color-channel-in-order-to-tint-the-image-with-the-desired-color.}{%
\paragraph{Answer: I could implement a function with p parameters for
each color channel in order to tint the image with the desired
color.}\label{answer-i-could-implement-a-function-with-p-parameters-for-each-color-channel-in-order-to-tint-the-image-with-the-desired-color.}}


    % Add a bibliography block to the postdoc
    
    
    
    \end{document}
